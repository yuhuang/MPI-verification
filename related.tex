\section{Related Works}
The POE algorithm is a DPOR (Dynamic Partial Order Reduction) algorithm \cite{DBLP:conf/popl/FlanaganG05} applied to MPI programs \cite{DBLP:conf/ppopp/VakkalankaSGK08}. It operates by postponing the cooperative operations for message passing in transit until each process reaches a blocking call. It then determines the potential matches of send and receive operations in runtime. In addition to program properties, it is able to check deadlocks that occur in MPI point-to-point communication.

MPI-Spin is integrated in the classic model checker, SPIN \cite{DBLP:journals/tse/Holzmann97}, for verifying MPI programs \cite{DBLP:conf/vmcai/Siegel07}. It generates a model of an MPI program and allows one to symbolically execute it. As such, violations such as deadlock are detected. Again, it does not scale to large programs with high message non-determinism.

To resolve the scalability problem, Vo. et al. use lamport clocks to update the auxiliary information via piggyback messages \cite{DBLP:conf/sc/VoAGSSB10, DBLP:conf/IEEEpact/VoGKSSB11}. While completeness is abandoned in their analysis, they claim this work is useful and efficient in practice. 

MCC is a model checker that systematically enumerates all message non-determinism in the MCAPI runtime under zero-buffer semantics \cite{DBLP:conf/fmcad/SharmaGMH09}. It employs dynamic partial order reduction to avoid enumerating redundant message orders. This work claims SMT technology is more efficient in practice in resolving message non-determinism. Elwakil et al. also use SMT techniques to reason about the program behavior in MCAPI domain \cite{DBLP:conf/issta/ElwakilY10}. State-based and order-based encoding techniques are both used. As discussed earlier, this technique fails to reason about the infinite buffer semantics and requires a precise match set which is non-trivial to compute beforehand.

There is a rich body of literature for SMT/SAT based Bounded Model Checking. 
TCBMC extends CBMC \cite{DBLP:conf/tacas/ClarkeKL04} to support concurrent C program verification \cite{DBLP:conf/cav/RabinovitzG05}. It bounds the number of context switches allowed among threads because it assumes that most bug patterns have only a few context switches. Especially, it assumes there is no nested lock-unlock pattern. As discussed earlier, this inspires our approach to encode MPI zero buffer semantics.
Burckhardt et al. present CheckFence prototype \cite{DBLP:conf/pldi/BurckhardtAM07} that exhaustively checks all executions of a test program by translating a program into SAT formulas. It increments the observations each time by adding more constraints to SAT formulas. 
%Dubrovin et al. give a method to translate an asynchronous system into a transition formula over three partial order semantics \cite{DBLP:journals/scp/DubrovinJH12}. The encoding adds constraints to compress the search space and decrease the bound on the program unwinding.