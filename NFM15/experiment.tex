\section{Experiments}
We compare the performance of our approach with two state-of-art MPI verifiers including ISP \cite{DBLP:conf/ppopp/VakkalankaSGK08,DBLP:conf/sbmf/SharmaGB12}, a dynamic analyzer and MOPPER \cite{DBLP:conf/fm/ForejtKNS14}, a SAT based tool. 
We set a series of experiments for five typical benchmark programs that are modified to be single-path. Assertions related to correct computation are manually inserted to each program. All the results show the comparison between zero buffer configuration and infinite buffer configuration. The initial program trace of our approach was generated by running MPICH \cite{mpich}, a publicly prevalent implementation of MPI standard, with fixed input and infinite buffer configuration. This program trace was encoded symbolically where each variable does not have a concrete value. A unique instance is generated for each write of a variable in the program computation (similar to the static single assignment form \cite{DBLP:journals/toplas/CytronFRWZ91}). Our encoding is resolved by the SMT solver Z3. MOPPER also needs a initial program trace with the same input data. This trace is generated by ISP. Since MOPPER is designed for deadlock checking, it does not encode any computation in a program. Therefore, the results only show the performance of MOPPER for zero buffer compatibility test. The experiments were run on a AMD A8 Quad Core processor with 6 GB memory running Ubuntu 14.04 LTS. We set a time limit of 30 minutes for each test. We abort the verification process if it did not complete within the time limit. 


\begin{savenotes}
\begin{table*}[t]
\begin{center}
\scriptsize
\caption{Tests on Selected Benchmarks}\label{table:benchmarks}
     \begin{threeparttable}
\begin{tabular}{|c|c|c|c|c|c|c||c|c||c|c||c|c|}
		\hline
         \multicolumn{7}{|c||}{Test Programs} & \multicolumn{2}{c||}{Our Method} & \multicolumn{2}{c||}{ISP} & \multicolumn{2}{c|}{MOPPER}  \\ \hline
          $Name$ & \#Procs & \#Calls&Match&B&Error & ZIC &Mem & Time &\#Runs&Time & Mem & Time\\ \hline
          \multirow{6}{*}{\textit{Monte}} & \multirow{2}{*}{4} & \multirow{2}{*}{35} &  \multirow{2}{*}{?} 
          												     & 0 & No & No\tnote{\textdagger} & 3.62 & 0.02s & 6 & 0.25s & 6.09 & $<$0.01s\\ \cline{5-13}
          						       &                            & &  &  $\infty$ & No & -- & 3.42 & 0.02s & 6 & 0.96s &  -- & --\\ \cline{2-13}
						       		& \multirow{2}{*}{8} & \multirow{2}{*}{75} &  \multirow{2}{*}{?} 
          												     & 0 & No & No\tnote{\textdagger} & 4.83 & 0.04s & $>$5K & TO & 11.28 & 0.02s\\ \cline{5-13}
          						       &                            & &  &  $\infty$ & No & -- & 4.34 & 0.04s & $>$5K & TO &  -- & --\\ \cline{2-13}
						              & \multirow{2}{*}{16} & \multirow{2}{*}{155} &  \multirow{2}{*}{?} 
          												     & 0 & No & No\tnote{\textdagger} & 8.97 & 0.29s & $>$5K & TO & 24.42 & 0.08s\\ \cline{5-13}
          						       &                            & &  &  $\infty$ & No & -- & 7.22 & 0.15s & $>$5K & TO & --  & --\\ \hline
						       \hline
						       
	   \multirow{6}{*}{\textit{Integrate}} & \multirow{2}{*}{8} & \multirow{2}{*}{36} &  \multirow{2}{*}{?} 
          												     & 0 & Yes & No & 4.71 & 0.08s & 1 & 0.15s & --  & -- \tnote{a}\\ \cline{5-13}
          						       &                            & &  &  $\infty$ & Yes & -- & 4.20 & 0.04s & 1 & 0.16s &  -- & --\\ \cline{2-13}
						       		& \multirow{2}{*}{10} & \multirow{2}{*}{46} &  \multirow{2}{*}{?} 
          												     & 0 & Yes & No & 5.39 & 0.08s & 1 & 0.16s & -- & -- \tnote{a}\\ \cline{5-13}
          						       &                            & &  &  $\infty$ & Yes & -- & 4.76 & 0.05s & 1 & 0.26s &  -- & --\\ \cline{2-13}
						              & \multirow{2}{*}{16} & \multirow{2}{*}{76} &  \multirow{2}{*}{?} 
          												     & 0 & Yes & No & 8.79 & 0.62s & 1 & 0.25s & -- & -- \tnote{a}\\ \cline{5-13}
          						       &                            & &  &  $\infty$ & Yes & -- & 7.50 & 0.32s & 1 & 0.54s & --  & --\\ \hline
						       \hline
						       
	    \multirow{6}{*}{\textit{Diffusion2D}} & \multirow{2}{*}{4} & \multirow{2}{*}{52} &  \multirow{2}{*}{?} 
          												     & 0 & No & Yes & 5.50 & 0.04s & 90 & 3.09s & 6.10 & 0.01s\\ \cline{5-13}
          						       &                            & &  &  $\infty$ & No & -- & 4.80 & 0.03s & 90 & 32.01s &  -- & --\\ \cline{2-13}
						       		& \multirow{2}{*}{8} & \multirow{2}{*}{108} &  \multirow{2}{*}{?} 
          												     & 0 & No & Yes & 11.94 & 0.22s & $>$9K & TO & -- & TO\\ \cline{5-13}
          						       &                            & &  &  $\infty$ & No & -- & 8.51 & 0.12s & $>$9K & TO &  -- & --\\ \cline{2-13}
						              & \multirow{2}{*}{16} & \multirow{2}{*}{228} &  \multirow{2}{*}{?} 
          												     & 0 & No & Yes & 30.68 & 1.25s & $>$10K & TO & -- & TO\\ \cline{5-13}
          						       &                            & &  &  $\infty$ & No & -- & 30.76 & 5.11s & $>$10K & TO & --  & --\\ \hline
						       \hline
						       
            \multirow{6}{*}{\textit{Router}} & \multirow{2}{*}{2} & \multirow{2}{*}{34} &  \multirow{2}{*}{?} 
          												     & 0 & No & Yes & 3.39 & 0.02s & 1 & 0.04s & -- & -- \tnote{a}\\ \cline{5-13}
          						       &                            & &  &  $\infty$ & No & -- & 3.37 & 0.02s & 60 & 13.24s &  -- & --\\ \cline{2-13}
						       		& \multirow{2}{*}{4} & \multirow{2}{*}{68} &  \multirow{2}{*}{?} 
          												     & 0 & No & Yes & 4.18 & 0.02s & 1 & 0.04s & -- & --\tnote{a}\\ \cline{5-13}
          						       &                            & &  &  $\infty$ & No & -- & 3.99 & 0.03s & $>$10K & TO &  -- & --\\ \cline{2-13}
						              & \multirow{2}{*}{8} & \multirow{2}{*}{136} &  \multirow{2}{*}{?} 
          												     & 0 & No & Yes & 5.17 & 0.04s & 1 & 0.15s & -- & --\tnote{a}\\ \cline{5-13}
          						       &                            & &  &  $\infty$ & No & -- & 5.06 & 0.05s & $>$11K & TO & --  & --\\ \hline
						       \hline
						       
	    \multirow{6}{*}{\textit{Floyd}} & \multirow{2}{*}{8} & \multirow{2}{*}{120} &  \multirow{2}{*}{?} 
          												     & 0 & No & No & 13.87 & 0.15s & $>$20K & TO & 18.05 & 0.27s\\ \cline{5-13}
          						       &                            & &  &  $\infty$ & No & -- & 12.14 & 0.12s & $>$20K & TO &  -- & --\\ \cline{2-13}
						       		& \multirow{2}{*}{16} & \multirow{2}{*}{256} &  \multirow{2}{*}{?} 
          												     & 0 & No & No & 21.58 & 0.26s & $>$20K & TO & 67.53 & 43.08s\\ \cline{5-13}
          						       &                            & &  &  $\infty$ & No & -- & 17.55 & 0.21s & $>$20K & TO &  -- & --\\ \cline{2-13}
						              & \multirow{2}{*}{32} & \multirow{2}{*}{528} &  \multirow{2}{*}{?} 
          												     & 0 & No & No & 252.97 & 439.89s & $>$20K & TO & 212.30 & 476.52s\\ \cline{5-13}
          						       &                            & &  &  $\infty$ & No & -- & 57.91 & 19.34s & $>$20K & TO & --  & --\\ \hline
         
\end{tabular}
\begin{tablenotes}
\item[\textdagger] MOPPER detects deadlock \item[a] SAT analysis is not required
\end{tablenotes}
     \end{threeparttable}
\end{center}
\end{table*}
\end{savenotes}

The results pertaining to the performance are documented in \tableref{table:benchmarks}. The column ``Match" records the approximated number of match resolutions. A program with a large number of match resolutions has a large degree of message non-determinism. The column ``Mem" records the memory cost by megabytes. The columns ``Time" for our approach and MOPPER are only for constraint solving. Note that our approach and MOPPER both spent less than one second to generate the trace and the encoding for every benchmark. We did not include the time for this process in the columns ``Time" because it is not interesting. The column ``\#Runs" for ISP is the number of program interleavings that ISP traverses before termination. The column ``Time" for ISP is the running time of dynamic analysis. The intuitive meaning of the symbol ``--" existing in many cells is ``not available".
 
%\begin{compactitem}

\textit{Monte} implements the Mento Carlo method to compute $\pi$ \cite{benchmark:mentoCarlo}. It uses one manger process and multiple worker processes to send messages back of forth. In addition, barrier operations are used to synchronize the program. 

\textit{Integrate} uses heavy non-determinism in message communication to compute an integral of $\sin$ function over the interval $[0, \pi]$ \cite{benchmark:fevs}. This benchmark also has a manger-worker pattern where the root process divides the interval in a certain number of tasks. It then distributes those tasks to multiple worker processes.
 
\textit{Diffusion2D} has an interesting computation pattern that uses barriers to ``partition" the message communication into several sections \cite{benchmark:fevs}. A message from a send can be only received in a common section. 

\textit{Router} is an algorithm to update routing tables for a set of nodes. Each node is in a ring and communicates only with immediate neighbors to update the tables. The program ends when all the routing tables are updated. 

\textit{Floyd} implements the shortest path algorithm for all the pairs of nodes \cite{DBLP:conf/ppopp/XueLWGCZZV09}. Each node communicates only with the immediate following neighbor.
%\end{compactitem}

%As shown in the prior work \cite{}, the size of match resolutions is the primary measure of scalability. 
The results show that solving our encoding by Z3 is highly efficient compared to ISP for assertion violation test and zero buffer compatibility test. For the benchmark programs such as \textit{Diffusion2D} and \textit{Floyd} that ISP does not terminate after traversing a large number of interleavings, our approach returns under a second in most tests. Even for the benchmark programs that ISP terminates after traversing only a small subset of all the interleavings, our approach is able to run slightly faster. Our approach is also faster than MOPPER for the benchmark programs with large degree of message non-determinism. If the number of match resolutions is low, our approach runs as fast as MOPPER does. Especially, MOPPER detects a deadlock for the program \textit{Monte} under zero buffer configuration. This deadlock is not caused by zero buffer incompatibility. As such, our approach is not able to detect it. Note that ISP misses this deadlock too. For the programs \textit{Integrate} and \textit{Router}, MOPPER does not launch a SAT analysis because it uses ISP as a preprocessor to initialize the input trace. If ISP detects assertion violation or deadlock in this preprocessing, MOPPER aborts the verification. 

%As before, we test both buffering settings for each benchmark program. The results show that the zero buffer encoding returns much faster than the infinite buffer encoding for all the benchmark programs. In particular, the test of \textit{Router} demonstrates that it is not compatible with zero buffer semantics. The size of match resolutions is still the primary measure of scalability. \textit{Matmat} is the easiest program to solve. This program has only a single match resolution for the SMT solver to consider thus returns quickly and uses little memory. The programs \textit{Mento Carlo}, \textit{Router}, \textit{Integrate} and \textit{Diffusion 2D} respectively increase the size of match resolutions. For example, even though \textit{Router} has 200 messages, it has less choices of match resolutions thus returns faster than \textit{Integrate} and \textit{Diffusion 2D}. 

%The benchmark suite demonstrates that an MPI program may have a large degree of message non-determinism in runtime. The number of match resolutions is the only deciding factor in scalability of our encoding. For example, even though \textit{Integrate} has more messages than the scalability test with 80 messages, it has less choices for resolving send-receive matches. Therefore, it can be solved with reasonable runtime. As such, the benchmark suite suggests that a program is able to complete in a reasonable amount of time if the number of match resolutions is under the boundary obtained in our scalability test. 