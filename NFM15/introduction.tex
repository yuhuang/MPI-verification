\section{Introduction}
Nowadays, message passing technology has become widely used in many fields such as medical devices and automobile systems. The Message Passing Interface (MPI) plays a significant role as a common standard. It is easy for a developer to implement a message passing scenario using MPI semantics, including:

\begin{compactitem}
\item zero buffer semantics (the message has no buffering) and infinite buffer semantics (the message is copied into a runtime buffer on the API call) \cite{DBLP:conf/fm/VakkalankaVGK09},
\item MPI point-to-point operations (e.g., send and receive), and
\item MPI collective operations (e.g., barrier and broadcast).
\end{compactitem}

A problem in any MPI program is zero buffer incompatibility. A zero buffer incompatible program deadlocks. Note that the zero buffer incompatibility is not equivalent to deadlock that may be caused by reasons other than buffering. The zero buffer incompatibility is essential to any MPI application since it is difficult for a developer to predict. 
This problem is also very difficult to verify because of the complicated MPI semantics.
In particular, the message passing may be non-deterministic such that a receive may match more than one send in the runtime system. Also, the MPI collective operations that synchronize the program may change how messages communicate. To address the problem of zero buffer incompatibility, this paper presents an algorithm that encodes a single-path MPI program into a Satisfiability Modulo Theories (SMT) problem \cite{barrett2008satisfiability}. This encoding is resolved by a standard SMT solver in such a way that the program is proved/disproved to be zero buffer compatible. This encoding is also adaptable to checking assertion violation caused by message non-determinism.

%Deadlock is another problem for an MPI program. It is caused by either the incompatibility of MPI semantics or improper implementations. The zero buffer compatibility is essential to any MPI application since it is difficult for a developer to predict this incompatibility. As a result, an execution on a zero buffer incompatible program may cause deadlock.
%A problem in any MPI program is assertion violation. It is often caused by message non-determinism that is difficult for a developer to test or debug without exploring the full program state space. During a message race, an unexpected message may be received leading to incorrect computation and violation of user provided assertions.  To address assertion violation and zero buffer incompatibility, this paper presents a method for encoding a single-path MPI program into a set of Satisfiability Modulo Theories (SMT) formulas \cite{barrett2008satisfiability}. This set of formulas is resolved by a standard SMT solver in such a way that assertion violation or zero buffer incompatibility is detected or the correctness of the program is proved. 

%The MPI semantics are complicated leading to two problems: message race and deadlock. The intermittent and unexpected message race is often caused by message non-determinism that is difficult for a developer to test or debug without exploring the full program state space. During a message race, an unexpected message may be received leading to incorrect computation and violation of user provided assertions. Deadlock is another problem for an MPI program. It is caused by either the incompatibility of MPI semantics or improper implementations. In particular, MPI applications may be not compatible with zero buffer semantics. It is difficult for a developer to know this incompatibility. As a result, an execution on a zero buffer incompatible program may cause deadlock. To address the problems of message race and zero buffer compatibility, this paper presents a method for resolving the message non-determinism in such a way that violations of user-provided assertions are detected. This method is also useful in detecting zero buffer compatibility of an MPI program. If the compatibility is proved, then the program is executable under zero buffer semantics. 

%MPI point-to-point operations (e.g., send and receive). The MPI semantics allow a receive to be matched with a send from a specific source or from any source. 
%
%. Such non-determinism often leads to intermittent and unexpected message race and is difficult for a developer to test and debug without exploring the full program state space. During a message race, an unexpected message may be received leading to incorrect computation and violation of user provided assertions. 
%
%%: collective operation
%MPI collective operations such as barriers and broadcasts may also affect MPI point-to-point communication. Any type of collective operations is used as a group. It synchronizes the program in such a way that each process waits on a specific location until all the operations in this group are completed. While those operations are convenient for synchronizing programs, they may interrupt how a message delivers. For instance, a message sent after a barrier may be not able to arrive at a specific receive if this receive has to be matched before the barrier. 
% 
%%: infinite buffer vs. zero buffer
%Message races are also affected by two runtime environments: infinite buffer semantics (the message is copied into a runtime buffer on the API call) and zero buffer semantics (the message has no buffering) \cite{DBLP:conf/fm/VakkalankaVGK09}. These two semantics may treat message communication very differently. Under infinite buffer semantics, for each message, a non-deterministic choice may be made to temporarily buffer or immediately receive the message once it is sent out. As such, more resolutions on matching a send operation and a receive operation for this message may exist in the runtime. In contrast, zero buffer semantics do not allow a message to be buffered in any process. A process waits until the pending message is received before proceeding. Consequently, there are fewer ways to resolve send-receive matches when using zero buffer semantics.

%zero buffer compatibility
%Because of the inflexible behavior that a message cannot be buffered in transit, many MPI applications are not compatible with zero buffer semantics. It is difficult for a developer to know this incompatibility. As a result, an execution on a zero buffer incompatible program may cause deadlock. 
%Therefore, the freedom of message passing can be balanced out in some way with respect to the runtime behaviors. 

%related works
Several solutions were proposed to verify MPI programs. The POE algorithm is capable of dynamically analyzing the behavior of an MPI program \cite{DBLP:conf/ppopp/VakkalankaSGK08}. This algorithm is implemented by a modern MPI verifier, ISP. As far as we know, there is no research proposed merely for zero buffer incompatibility. Though the works on MPI deadlock are also capable of detecting zero buffer incompatibility, they suffer from the scalability problem \cite{DBLP:conf/sbmf/SharmaGB12,DBLP:conf/fm/ForejtKNS14}. In particular, the algorithm MSPOE is an extension of POE. It is able to detect deadlock in an MPI program \cite{DBLP:conf/sbmf/SharmaGB12}. %Similar to the POE algorithm, MSPOE does not scale to the programs with large degree of message communication. 
Forejt et al. proposed a SAT based approach to detect deadlock in a single-path MPI program \cite{DBLP:conf/fm/ForejtKNS14}. 
%However, since the size of their encoding is cubic, checking large programs is time consuming. 
In the context of the Multicore Communications API (MCAPI) \cite{mcapi}, another message passing API standard, Elwakil et al. encoded an MCAPI program execution into an SMT problem \cite{DBLP:conf/issta/ElwakilY10,DBLP:conf/atva/ElwakilYW10}. This approach assumes that potential send-receive matches are given. However, the problem of generating such a set is NP-Complete.

%fails for two reasons. First, the encoding does not support infinite buffer semantics. Second, it assumes that potential send-receive matches are given as an input to the encoding. However, the problem of generating such a set is NP-Complete.



%Sharma et al. proposed the first push button model checker -- MCC \cite{DBLP:conf/fmcad/SharmaGMH09}. It indirectly controls the MCAPI runtime to verify MCAPI programs under zero buffer semantics. An obvious drawback of this work is its inability to analyze infinite buffer semantics which is known as a common runtime environment in message passing. A key insight, though, is the direct use of match pairs -- couplings for potential sends and receives.



%Further, their rules for zero buffer encoding is too simple, therefore is hard to enforce the zero buffer semantics into SAT formulas. 

%our solution: SMT, statically encode MPI semantics
This paper presents a new algorithm that encodes a single-path MPI program into an SMT problem. This algorithm also asks for a match pair set as input. The preprocessing, however, only needs to over-approximate the match pairs in quadratic time complexity. Before discussing how this algorithm is capable of detecting zero buffer incompatibility, this paper needs to present in detail the complete list of encoding rules for several MPI operations, including a few rules that are trivial to define.  In particular, the MPI non-deterministic point-to-point communication is similar to the message passing semantics in MCAPI. As such, this paper adapts the existing encoding rules for MCAPI defined in prior work \cite{DBLP:conf/kbse/HuangMM13}. This paper also presents how to encode deterministic receive operations and collective operations, which are essential to MPI semantics, into a set of SMT formulas. The formula size is quadratic. Note that the prior work also provides a list of non-intuitive and complicated zero buffer encoding rules. However, these rules are only useful for manually encoding the zero buffer semantics. The new zero buffer encoding in this paper considers only the schedules that strictly alternate sends and receives, therefore, is correct and intuitive, and is easy to build automatically. The use of strict alternating is able to cover the message communication that may occur in any execution under zero buffer semantics. This strategy is inspired by the Threaded-C Bounded Model Checking (TCBMC) that assumes each lock operation and its paired unlock operation is ordered alternatingly in any execution \cite{DBLP:conf/cav/RabinovitzG05}. 

To summarize, the main contributions of this paper include,
%Contribution:
\begin{itemize}
\item a new zero buffer encoding with strict alternating of sends and receives that is capable of detecting zero buffer incompatibility, 
\item a new encoding algorithm that supports MPI deterministic point-to-point communication and MPI collective communication, and
\item a set of benchmarks that demonstrate the new encoding is more efficient than two state-of-art MPI verifiers.
\end{itemize}

%section Organization
The rest of the paper is organized as follows: Section $2$ discusses the MPI semantics given an example; Sections $3$ and $4$ present a few trivial rules for encoding MPI operations, including the summarization of the prior work \cite{DBLP:conf/kbse/HuangMM13} in section $3$ and the rules for MPI deterministic operations and collective operations in section $4$; Based on these rules, section $5$ discusses the new zero buffer encoding and how it is able to check zero buffer incompatibility; Section $6$ gives the experiment results; Section $7$ discusses the related work; and Section $8$ discusses the conclusion and future work.

\examplefigone


           