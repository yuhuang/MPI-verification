%Asynchronous message passing systems are fast becoming a common means for communication between devices.
%
A prevalent asynchronous message passing standard is the Message Passing Interface (MPI). 
%
As many MPI applications imperatively request, two runtime semantics, zero buffer (message has no buffering) and infinite buffer (message is copied into a runtime buffer on the API call) are supported.
%
A problem in any MPI program, intended or otherwise, is zero buffer incompatibility. A zero buffer incompatible MPI program deadlocks.
%
This problem is difficult to predict because a developer does not know if the deadlock is based on buffering setting or improper implementation. 
%
This paper presents an algorithm that encodes a single-path MPI program as a Satisfiability Modulo Theories (SMT) problem, which if satisfiable, yields a feasible schedule, such that it proves this program is zero buffer compatible. This encoding is also adaptable to checking assertion violation for correct computation.
%
To support MPI semantics, this algorithm correctly defines the point-to-point communication and collective communication with respective rules for both infinite buffer semantics and zero buffer semantics. 
%
The novelty in this paper is considering only the schedules that strictly alternates sends and receives leading to an intuitive zero buffer encoding.
%
This paper proves that the set of all the strictly alternating schedules is capable of covering the message communication that may occur in any execution under zero buffer semantics. 
%
% message non-determinism is notorious for its difficulty to test or debug without exploring the full program state space. 
%
%As both runtime semantics, infinite buffer (message is copied into a runtime buffer on the API call) and zero buffer (message has no buffering), are imperatively requested by many MPI applications, both should be supported in a formal analysis.
%
%A  problem in the buffering setting is the zero buffer compatibility. A deadlock occurs by running any zero buffer incompatible MPI program.
%
%This paper presents an algorithm that encodes an MPI execution as a Satisfiability Modulo Theories (SMT) problem, which if satisfiable, yields a feasible schedule on the same trace, such that it resolves message non-determinism in a way that it now fails user provided assertions.
%
%This algorithm correctly defines MPI point-to-point communication and MPI collective communication with respective rules for both infinite buffer semantics and zero buffer semantics. 
%
%The zero buffer encoding is further capable of precisely detecting zero buffer compatibility for an MPI program. 
%
%If the compatibility is proved, the MPI program is executable under zero buffer semantics.
%
Experiments demonstrate that the SMT encoding is correct and highly efficient for a set of benchmarks compared with two state-of-art MPI verifiers. 
%
%The encoding scales well for programs with a high level of message non-determinism in how sends and receives may potentially match.