\section{Deadlock in MPI}
The complexity proof is on a
new decision problem: \emph{Verifying Deadlock Violation in an MPI program}
(VDMPI).
\begin{definition}
\textbf{Verifying Deadlock Violation in an MPI program}.
\begin{compactdesc}
\item{INSTANCE}: A set of constants $D$, a set of variables $X$, a finite set of processes $P$, and a
  set of partial orders $\prec$ over send, receive, assert and
  collective operations.
\item{QUESTION}: Is there a coherent schedule $S$ for the operations of $P$ that a deadlock occurs where some send/receive
is orphaned?
\end{compactdesc}
\label{def:np1}
\end{definition}

The VDMPI problem is NP-Complete. The proof is reducing the problem from SAT in polynomial time. In particular, given an instance $Q$ of SAT consisting of a set of variables $U$ and a set of clauses $C$ over $U$, an instance $V$ of VDMPI is constructed such that $V$ has a coherent schedule $S$ for the operations of $P$ that a deadlock occurs if and only if a satisfying assignment for $Q$ exists. Coherent in this context indicates the schedule is allowed by the MPI semantics.

The reduction is illustrated in \figref{fig:vdmpi}. The wait functions are omitted because each of them follows its send/receive directly in the figure. The subscript notations of all sends/receives are omitted as well because they are redundant. The values sent are added to the sends as they are essential to the reduction. 

Intuitively, the reduction relies on message non-determinism to evaluate the value of each variable in SAT. The process $\mathit{p_C}$ repeatedly sends the signal $d_s$ to two processes, $\mathit{p_{d_0}}$ and $\mathit{p_{d_1}}$. After the signals are received by $\mathit{p_{d_0}}$ and $\mathit{p_{d_1}}$, they sends the false value $\mathit{d_0}$ or the true value $\mathit{d_1}$ respectively to $\mathit{p_{C}}$ for evaluating a variable in SAT. The message arrivals from $\mathit{p_{d_0}}$ and $\mathit{p_{d_1}}$ are racing due to semantics of wildcard receives in $\mathit{p_C}$. Therefore, the valuation of the variables in SAT can be non-deterministically $\mathit{d_0}$ or $\mathit{d_1}$ that is received by $\mathit{p_C}$. The new signals are sent only when the current variable in $\mathit{p_C}$ is assigned a value. Therefore, the valuation of the new variable is not resolved until the evaluation of the current variable is finished.

The valuation of each variable in SAT is resolved eventually by sequentially requesting and receiving values in $\mathit{p_C}$. After that, $\mathit{p_C}$ evaluates $Q$, the conjunction of all clauses $C$ in SAT, as a logic condition in a if statement. Since each variable in SAT is assigned a value, $Q$ is evaluated sequentially. If a satisfying assignment exists for $Q$, then a wildcard receive is issued by $\mathit{p_C}$. This receive can not match any sends in the program. Therefore, this orphaned receive leads to a deadlock in this coherent schedule. 
 
\begin{figure}
\begin{center}
\setlength{\tabcolsep}{2pt}
\begin{tabular}[t]{|l|l|l|}
\hline
\multicolumn{3}{|l|}{\textbf{SAT:} $\mathit{U\equiv\{u_0,u_1,...,u_m\}}$}\\
              \multicolumn{3}{|l|}{$\mathit{C\equiv\{c_0,c_1,...,c_n\}}$}\\
              \multicolumn{3}{|l|}{$\mathit{Q\equiv\{c_0\wedge c_1\wedge ...\wedge c_n\}}$}\\
\hline
\multicolumn{3}{|l|}{\textbf{VAMPI:} $\mathit{P\equiv\{p_{d_0},p_{d_1},p_{C}\}}$}\\
                \multicolumn{3}{|l|}{$\mathit{X\equiv\{u_0,...,u_{m+1},g_0,g_1\}}$}\\
                \multicolumn{3}{|l|}{$\mathit{D\equiv\{d_0,d_1,d_s\}}$}\\
\hline
$p_{d_0}$ & $p_{d_1}$ & $p_C$ \\
\hline
$R(g_{0},*)$   & $R(g_{1},*)$   & $S(d_s,h_{d_0})$ \\
$S(d_{0},h_C)$ & $S(d_{1},h_C)$ & $S(d_s,h_{d_1})$ \\
              &                & $R(u_0,*)$        \\
              &                & $R(u_0,*)$        \\
\hline
$R(g_{0},*)$  &  $R(g_{1},*)$   & $S(d_s,h_{d_0})$ \\
$S(d_{0},h_C)$&  $S(d_{1},h_C)$ & $S(d_s,h_{d_1})$ \\
              &                & $R(u_1,*)$       \\
              &                & $R(u_1,*)$       \\
\hline
$\ldots$      &  $\ldots$      & $\ldots$                \\
\hline
              &                & \textit{if}($c_0\wedge c_1\wedge ...\wedge c_n$) \\
              &                & $R(u_{m+1},*)$ \\
\hline
\end{tabular}
\end{center}
\caption{General SAT to VDMPI reduction}
\label{fig:vdmpi}
\end{figure}

\begin{lemma} \label{lem:sat}
$S$ is a coherent schedule for $P$ that a deadlock occurs if and only if $Q$ is satisfiable. 
\end{lemma}
\begin{proof}
\textbf{Coherent schedule for V that a deadlock occurs implies Q is satisfiable}: proof by
contradiction. Assume that $Q$ is unsatisfiable even though there is a
coherent schedule $S$ for $V$ that a deadlock occurs that some send/receive is orphaned. The
reduction in \figref{fig:vdmpi} considers all truth values of the
variables in $Q$, over every combination, by virtue of the
non-determinism, and then issues a wildcard receive if each of the clauses in
$Q$ is true. The complete set of possibilities is realized by sending in
parallel from $\mathit{p_{d_0}}$ and $\mathit{P_{d_1}}$ the two truth valuations for a
given variable to $\mathit{p_C}$. As these messages may be received in any
order, each variable may assume either truth value. Further, each
variable resolved is an independent choice so all combinations of
variable valuations must be considered.  This fact is a contradiction
to the assumption of $Q$ being unsatisfiable as the same truth values
that meet the assertions would be a satisfying assignment in $Q$.

\noindent \textbf{Q is satisfiable implies feasible schedule for V}:
the proof is symmetric to the previous case and proceeds in a like
manner.
\end{proof}

\begin{theorem}[NP-complete]
VDMPI is NP-complete.
\end{theorem}
\begin{proof}
\noindent\textbf{Membership in NP}: a certificate is a schedule
that a deadlock occurs that some send/receive is orphaned. The schedule is
linearly scanned with the process histories and checked that it does not
violate MPI semantics. We construct an operational model of MPI semantics that obtains every possible schedule that can be resolved by a program execution. The schedule mentioned above can be obtained by operating this model with the process histories. 
%The complexity is linear in the size of the schedule.

\noindent\textbf{NP-hard}: polynomial reduction from SAT. The
correctness of the reduction is established by \lemmaref{lem:sat}.
The remainder of the proof is the complexity of the reduction. There
are two processes to send values $d_0$ and $d_1$ upon request. For each
variable $u_i \in U$, each of these processes, $d_0$ and $d_1$, needs two
operations: one to synchronize with $h_C$ and another to send the
value: $O(2 * 2 * |U|)$. The task $h_C$ must request values from
$h_{d_0}$ and $h_{d_1}$ and then receive both those values; it must do
this for each variable: $O(2 * 2 * |U|)$.  Once all the values
are collected, it must then evaluate each clause in the if condition: $O(|C|)$. As every
term is linear, the reduction is linear.
\end{proof}