\section{Experiments}
We launch three series of experiments to test our new encoding: a correctness test on the running examples presented in this paper, a scalability test on the effects of message non-determinism, and a typical benchmark test on several programs. All the results show the comparison between the zero-buffer encoding and the infinite-buffer encoding.  Those experiments were run on a 2.40 GHz Intel Quad Core processor with 8 GB memory running Windows 7. We set a time limit of 2 hours for each test. We abort the verification process if it did not complete within the time-limit. 

The initial program trace is generated by running MPICH \cite{}, a publicly prevalent implementation of MPI standard, with fixed input. The experiments only consider one path of the control flow through the program. Complete coverage of the program for verification purposes would need to generate input to exercise different control flow paths. 

\subsection{Correctness Test}
We test our running examples in \figref{mpi} and \figref{mpi_barrier} with zero-buffering semantics and infinite-buffering semantics. The purpose of this test is to demonstrate that our encoding is capable of detecting assertion violations in an execution trace.

\begin{table}[t]
\begin{center}
\scriptsize
\caption{Tests on Programs in \figref{fig:mpi} and \figref{fig:mpi_barrier}}
\begin{tabular}{|c|c|c|c|c|c|}
		\hline
         \multicolumn{3}{|c|}{Test Programs} & \multicolumn{3}{|c|}{Performance} \\ \hline
          $Source$&Matches&Buffering&Result&Time(s)&Memory(MB) \\ \hline
          \figref{fig:mpi} & 1 & 0 & unsat & 0.03 & 2.19 \\
          	     & 2 & $\infty$ & sat & 0.13 & 2.18 \\ \hline
          \figref{fig:mpi_barrier} & 1 & 0 & unsat & 0.03  & 2.19 \\
           & 1 & $\infty$ & unsat & 0.07 & 2.19 \\
          \hline
		\end{tabular}
\end{center}
\end{table}

\subsection{Scalability Test}


\begin{table}[t]
\begin{center}
\scriptsize
\caption{Scalability Test Results}
\begin{tabular}{|c|c|c|c|c|}
		\hline
         \multicolumn{3}{|c|}{Test Programs} & \multicolumn{2}{|c|}{Performance} \\ \hline
          $N$ & Feasible Sets & Buffering  & Time(hh:mm:ss) & Memory(MB) \\ \hline
          30 & 30!($\sim$3E32) & 0 & 00:00:01 & 15.17\\
               & & $\infty$ & 00:00:27 & 19.53 \\ \hline
          40 & 40!($\sim$8E47) & 0 & 00:00:08 & 43.81\\
               &  & $\infty$ & 00:03:31 & 48.41 \\ \hline
          50 & 50!($\sim$3E64) & 0 & 00:00:28 & 78.03\\
               & & $\infty$ & 00:12:43 & 91.26\\ \hline
          60 & 60!($\sim$8E81) & 0 & 00:08:13 & 210.82\\
               &  & $\infty$ & 00:36:56 & 174.83\\ \hline
          70 & 70!($\sim$1E100) & 0 & 00:12:02 & 320.49\\
               & & $\infty$ & 01:32:17 & 312.35\\ \hline
          80 & 80!($\sim$2E152) & 0 & 01:33:08 & 728.97\\
               & & $\infty$ & -- & --\\ \hline         
		\end{tabular}
\end{center}
\end{table}

\subsection{Benchmark Test}

\begin{table}[t]
\begin{center}
\scriptsize
\caption{Tests on Selected Benchmarks}
\begin{tabular}{|l|c|c|c|c|}
		\hline
         \multicolumn{3}{|c|}{Test Programs} & \multicolumn{2}{|c|}{Performance} \\ \hline
          $Name$&\# Mesg&Buffering&Time(hh:mm:ss)&Memory(MB) \\ \hline
          \textit{Matmat} &  & 0 & & \\
          	      & &  $\infty$ & & \\ \hline
	 \textit{Mento} & 49 & 0 & 00:00:02 & 21.76\\
          \textit{Carlo}    & & $\infty$ & 00:00:02 & 18.08 \\ \hline
	 \textit{Router}   & 200 & 0 & 00:00:02 & 16.98\\
          	      & & $\infty$ & $<$00:00:01 & 31.30 \\ \hline
	 \textit{Integrate}  & 123 & 0 & 00:00:45 & 135.06 \\
          	      & & $\infty$ & 00:08:07 & 238.68\\ \hline
	  \textit{Diffusion}  & 54 & 0 & 00:00:09 & 44.01 \\
           \textit{2D} &  & $\infty$ & 00:09:15 & 92.08\\ \hline          
          		\end{tabular}
\end{center}
\end{table}


