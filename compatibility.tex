\section{Zero Buffer Compatibility}
The correctness discussed above further demonstrates the possibility to precisely check zero buffer compatibility of an MPI program. Notice that the input trace of our encoding is generated from infinite buffer semantics, the match pairs may not be resolved under zero buffer semantics. For example, \figref{fig:compatible} shows a simple scenario that issues two consecutive sends on each process. 

\begin{figure}[h]
\[
\begin{array}{l|l}
\;\;\;\;\;\;\;\;\mathtt{Process\ 0}\;\;\;\;\;\;\;\; & \;\;\;\;\;\;\;\; \mathtt{Process\ 1}\;\;\;\;\;\;\;\; \\
\hline
\;\;\;\;\;\;\;\;\mathtt{S(to\ P1, ``1",\&h1)}\;\;\;\;\;\;\;\; & \;\;\;\;\;\;\;\; \mathtt{S(to\ P0,``2"\&h2)}\;\;\;\;\;\;\;\; \\
\;\;\;\;\;\;\;\;\mathtt{W(\&h1)}\;\;\;\;\;\;\;\; & \;\;\;\;\;\;\;\; \mathtt{W(\&h2)}\;\;\;\;\;\;\;\; \\
\;\;\;\;\;\;\;\;\mathtt{S(to\ P1,``3",\&h3)}\;\;\;\;\;\;\;\; & \;\;\;\;\;\;\;\; \mathtt{S(to\ P0,``4",\&h4)}\;\;\;\;\;\;\;\; \\
\;\;\;\;\;\;\;\;\mathtt{W(\&h3)}\;\;\;\;\;\;\;\; & \;\;\;\;\;\;\;\; \mathtt{W(\&h4)}\;\;\;\;\;\;\;\; \\
\;\;\;\;\;\;\;\;\mathtt{\cdots}\;\;\;\;\;\;\;\; & \;\;\;\;\;\;\;\; \mathtt{\cdots}\;\;\;\;\;\;\;\; 
\end{array}
\]
\caption{An Example of Zero Buffer Compatibility} \label{fig:compatible}
\end{figure}

There is no way to execute the scenario in \figref{fig:compatible} under zero buffer semantics because the first send on each process never returns leading to a deadlock of the program. We are able to check zero buffer compatibility of this scenario by merely encoding and resolving the partial order constraints that are allowed by zero buffer semantics. We simply remove the constraints for user-provided assertions because they are irrelevant for this purpose. If the encoding is satisfiable, the compatibility of this scenario is proved; otherwise, it can not be executed. 