\section{SMT Encoding}
The SMT encoding is generated from 1) an execution trace of an MPI program that includes a sequence of events, i.e., point-to-point operations, collective operations, control flow assumptions and property assertions; and 2) a set of possible match pairs for message communication. Intuitively, a match pair is a coupling of a send and a receive. Taking \figref{fig:mpi} as an example, each receive on process $0$ may be matched with a send from process $1$ or process $2$. The direct use of match pair, rather than a state based approach \cite{elwakil:padtad10} or a indirect use of match pair in an order based encoding \cite{elwakil:atva10}, is easy to reason about message communication. It also uses \textit{happens-before} relation stemming from the program order for each process model and each concurrency relation. A modern SMT solver, such as Yice \cite{dutertre:CAV06} or Z3 \cite{demoura:tacas08}, is asked to resolve this encoding. A resolved satisfying schedule is guaranteed to be feasible with a violation of user provided assertions in the system runtime. If no satisfying schedule exists, the correctness of the program is also proved (no violation of the assertion) in all possible executions. 

According to the discussion earlier, the MPI behavior can be affected by two runtime settings: infinite buffer semantics and zero buffer semantics. The infinite buffer semantics behave differently from the zero buffer semantics in how message communicates. Therefore, the encoding should use different rules to support both settings. 

This section first discusses the algorithm to encode infinite buffer semantics, including the rules for point-to-point operations, collective operations and how they are constrained in program order. It then discusses how to adjust this encoding to zero buffer semantics by adding a few new rules.

\subsection{Partial Order}
The encoding needs to express the partial order imposed by the MPI semantics as SMT constraints. Such a partial order needs to bind an ``order" variable (\defref{def:order}) to each event in the program. The pertinent ``order" variables of two events are used to constrain their partial order as a \emph{happens-before} relation (\defref{def:happens-before}).

\begin{definition}[Order]\label{def:order}
The order of an event $\mathtt{e}$, denoted as $\mathit{order}_\mathtt{e}$, is constrained as an integer.
\end{definition}

\begin{definition}[Happens-Before]\label{def:happens-before}
The \emph{Happens-Before} $(\mathtt{HB})$ relation, denoted as
$\mathit{order}_\mathtt{e1} \mathrm{\prec_\mathtt{HB}} \mathit{order}_\mathtt{e2}$, is a partial order over two events, $\mathtt{e1}$ and $\mathtt{e2}$ respectively, that constrains the logic formula $\mathit{order}_\mathtt{e1} <  \mathit{order}_\mathtt{e2}$. 
\label{def:hb}
\end{definition}

In \defref{def:happens-before}, if $\mathtt{e1}$ must happen before $\mathtt{e2}$ in any execution, the encoding needs to add $\mathit{order}_\mathtt{e1} \mathrm{\prec_\mathtt{HB}} \mathit{order}_\mathtt{e2}$ in constraint. The goal of an SMT solver is to evaluate all the ``order" variables in such a way that satisfies all the $\mathtt{HB}$ relations. Based on the $\mathtt{HB}$ relation, the encoding further defines a few rules to constrain the program order. 

\subsection{Point-To-Point Communication}
%:Definitions
Before discussing those rules, it is necessary to define MPI point-to-point operations as SMT constraints. The MPI point-to-point communication consists of two basic  primitives: send and receive. The encoding defines send (\defref{def:snd}) and receive (\defref{def:rcv}) as a set of variables in order to constrain the concurrent behavior. Some variables, such as the endpoint information, are already evaluated from the existing trace. The other variables, such as the order information, need to be resolved by an SMT solver. 
 
\begin{definition}[Send] \label{def:snd}
A send operation $\mathtt{S}$, is a four-tuple of variables:
\begin{compactenum}
\item $M_\mathtt{S}$, the order of the matching receive event;
\item $\mathit{order}_\mathtt{S}$, the order of the send;
\item $e_\mathtt{S}$, the destination endpoint; 
\item $src_\mathtt{S}$, the source endpoint; and
\item $\mathit{value}_\mathtt{S}$, the transmitted value.
\end{compactenum}
\end{definition}

As shown in \defref{def:snd}, the encoding of a send has a few evaluated variables. Those variables include the source endpoint, the destination endpoint and the transmitted value. As for the free variables, the encoding records the event orders for itself and the matching receive as $\mathit{order}_\mathtt{S}$ and $M_\mathtt{S}$ respectively. In particular, $M_\mathtt{S}$ is used to bind a specific receive if a message from $\mathtt{S}$ flows across this receive. Such a send-receive match is only resolved dynamically by an SMT solver. As long as a value is assigned to $M_\mathtt{S}$, $\mathtt{S}$ is not allowed to match with another receive (otherwise it is a false constraint).

\begin{definition}[Wait] \label{def:wait}
The occurrence of a wait operation, \texttt{W}, is captured by a
single variable, $\mathit{order}_\mathtt{W}$, that constrains when
the wait occurs.
\end{definition}

In MPI semantics, a non-blocking send is paired with a wait operation that confirms the message has been copied out of the send buffer. The wait for a send does not affect message communication mainly because it is not able to detect a matching receive when it returns. As such, the encoding simply ignore the wait for a send. In contrast,  a wait for a receive is able to confirm message communication. The encoding defines such a wait operation in \defref{def:wait}. The wait for a receive witnesses the message is copied into the receive buffer. So a matching send must be detected when the wait returns. This behavior confirms two things: 1) a receive is issued before its wait; and 2) a  send is issued before the wait of a matching receive. The encoding constrains this behavior with a few rules discussed later. Further, the MPI semantics allow a single wait to witness the completion of one or more receives due to the message non-overtaking property. Such a wait is the nearest-enclosing wait. 

\begin{definition}[Nearest-Enclosing Wait] \label{def:nw}
A wait that witnesses the completion of a receive by indicating that
the message is delivered and that all the previous receives in the
same task issued earlier are complete as well.
\end{definition}
%: an example to show nearest-encolsing wait

The encoding requires that every receive has a nearest-enclosing wait so a match pair decision can be made at this wait. Based on this requirement, a receive should include the nearest-enclosing wait in its variables (\defref{def:rcv}).

\begin{definition}[Receive] \label{def:rcv}
A receive operation $\mathtt{R}$ is modeled by a five-tuple of variables:
\begin{compactenum}
\item $M_\mathtt{R}$, the order of the matching send event;
\item $\mathit{order}_\mathtt{R}$, the order of the receive;
\item $e_\mathtt{R}$, the destination endpoint;
\item $src_\mathtt{R}$, the source endpoint;
\item $\mathit{value}_\mathtt{R}$, the received value; and,
\item $\mathit{nw}_\mathtt{R}$, the order of the nearest enclosing wait.
\end{compactenum}
\end{definition}

The encoding of a receive has a variable for the destination endpoint that is evaluated from the existing trace. It defines two free variables $\mathit{order}_\mathtt{R}$ and $M_\mathtt{R}$, for the event orders of itself and the matching send, respectively. The transmitted (received) value is not known until a send is matched. Interestingly, the source endpoint for a receive may or may not be known depending which receive it defines. The MPI semantics support both deterministic receive with a constant source endpoint and wildcard receive with a uncertain source endpoint. The nearest-enclosing wait, as discussed earlier, is used to make a match pair decision. 

The encoding uses match pairs directly to express the message communication. Intuitively, \defref{def:match} constrains the event order, the endpoints and the transmitted value are matched for $\mathtt{R}$ and $\mathtt{S}$. Also, $\mathtt{S}$ should be issued before the nearest-enclosing wait of $\mathtt{R}$. 

\begin{definition}[Match Pair] \label{def:match}
A match pair, $\langle\mathtt{R}, \mathtt{S}\rangle$, for a receive
$\mathtt{R}$ and a send $\mathtt{S}$ corresponds to the constraints:
\begin{compactenum}
\item $M_{\mathtt{R}} = \mathit{order}_{\mathtt{S}}$
\item $M_{\mathtt{S}} = \mathit{order}_{\mathtt{R}}$
\item $e_{\mathtt{R}} = e_{\mathtt{S}}$
\item $src_\mathtt{R} = src_\mathtt{S}$
\item $\mathit{value}_{\mathtt{R}} = \mathit{value}_{\mathtt{S}}$ and
\item $\mathit{order}_{\mathtt{S}}\ \mathrm{\prec_\mathtt{HB}}\ \mathit{nw}_{\mathtt{R}}$
\end{compactenum}
\end{definition}

The use of a set of match pairs may reflect all possible ways for the message communication. There may exist several match pairs for a single receive indicating more messages may flow across this receive. However, there is no way to receive all the messages in a single execution, even though all the match pairs should be considered. Therefore, the encoding does not combine these match pairs in a single conjunction. Instead, it constrains them for a single receive in a disjunction (\defref{def:receive_match}). As the $M$ values for both a receive and a send are deterministic once they are matched, only one match pair can be used in a final resolution.

\begin{definition}[Receive Matches] \label{def:receive_match}
For each receive $\mathtt{R}$, if $\langle\mathtt{R},
\mathtt{S}_0\rangle$ through $\langle\mathtt{R}, \mathtt{S}_n\rangle$
are match pairs, then $\bigvee_{i}^{n} \langle\mathtt{R},
\mathtt{S}_i\rangle$ is used as an SMT constraint.
\end{definition}


\subsection{Collective Communication}
Collective communication is also significant in MPI semantics. There are various collective operations used for this communication.  Take barriers as an example, they are issued in a group. Each process waits on a barrier without proceeding until all group members are completed. The other collective operations synchronize the program similarly except that they additionally address the tasks of internal message communication and/or computation. Those tasks are not interrupted by MPI point-to-point communication, and vice versa. According to this fact, the algorithm only needs to consider how to constrain the concurrent behavior, but simply constrains each task as an assertion. As such, we take barriers in the following discussion as it is merely used for program synchronization. Because of the concurrent behavior, an operation issued before the barriers is not able to interleave any operations issued after the barriers. The encoding defines the barrier in \defref{def:barrier}. 

\begin{definition}[Barrier]\label{def:barrier}
The occurrence of a barrier operation, \texttt{B}, is captured by a
single variable, $\mathit{order}_\mathtt{B}$, that constrains when a group of barriers $\{B_0, B_1, ..., B_n\}$ are matched.  
Each barrier $B_i, i\in{0 ... n}$, is issued by process $i$. 
\end{definition}

\begin{figure}[h]
\[
\begin{array}{l|l}
\;\;\;\;\;\;\;\;\mathtt{Process\ 0}\;\;\;\;\;\;\;\; & \;\;\;\;\;\;\;\; \mathtt{Process\ 1}\;\;\;\;\;\;\;\; \\
\hline
\;\;\;\;\;\;\;\;\mathtt{\underline{B(comm)}}\;\;\;\;\;\;\;\; & \;\;\;\;\;\;\;\; \mathtt{R(from\ P0,A\&h2)}\;\;\;\;\;\;\;\; \\
\;\;\;\;\;\;\;\;\mathtt{S(to\ P1,``1",\&h1)}\;\;\;\;\;\;\;\; & \;\;\;\;\;\;\;\; \mathtt{\underline{B(comm)}}\;\;\;\;\;\;\;\; \\
\;\;\;\;\;\;\;\;\mathtt{W(\&h1)}\;\;\;\;\;\;\;\; & \;\;\;\;\;\;\;\; \mathtt{W(\&h2)}\;\;\;\;\;\;\;\; \\
\end{array}
\]
\caption{An Example of Message Communication with Barriers} \label{fig:mc_barrier1}
\end{figure}

As for the affection on message communication, the barriers may not prevent a send from matching a receive in some case, even though their issuing order is affected. As an example, executing the program in \figref{fig:mc_barrier1} may flow the message ``$1$" across $\mathtt{R}$ even though $\mathtt{R}$ is issued before the barriers and $\mathtt{S}$ is issued after the barriers. However, if the program issues $\mathtt{W(\&h2)}$ before barriers, $\mathtt{R}$ has to be completed before barriers meaning a message has to be delivered. In this situation, the message communication is affected. The algorithm further defines the \textit{nearest-enclosing barrier} (\defref{def:nb}) to constrain this affection.

\begin{definition}[Nearest-Enclosing Barrier]\label{def:nb}
For a process $i$, a receive \texttt{R} has a nearest-enclosing barrier \texttt{B} if and only if
\begin{compactenum}
\item the nearest-enclosing wait \texttt{W} of \texttt{R} is issued before $B_i\in B$, and
\item there does not exist any receive \texttt{R'} that has its nearest-enclosing wait \texttt{W'} issued after \texttt{W} and issued before $B_i$.
\end{compactenum}
\end{definition}


\subsection{Assumptions and Assertions}
Except the point-to-point and the collective operations, the assumptions and the assertions should be constrained as well. An assumption asks any feasible execution to obey the imposed control flow path. So it can be viewed as a inviolated truth in any execution. The encoding constrains this truth as an SMT assertion.
\begin{definition}[Assumption] \label{def:assm}
Every assumption $\mathtt{A}$ is added as an SMT assertion.
\end{definition}

An assertion checks if a feasible execution holds the property that is interesting to the user. The goal of our encoding is to detect a hidden assertion violation in any execution. Therefore, the \textit{negation} of this property should be encoded so any satisfying assignment corresponds to a witness of violation. 

\begin{definition}[Property Assertion] \label{def:assert}
For every property assertion $\mathtt{P}$, $\neg \mathtt{P}$ is added as
an SMT assertion.
\end{definition}

\subsection{Program Order}
The encoding thus far defines a few terms that are used to constrain the program order for infinite buffer semantics. The program order can be added in seven steps: we must ensure that two sends to a common endpoint must be ordered on each process (step 1); similar to the receives (step 2); receives happen before their nearest-enclosing wait (step 3); the nearest-enclosing wait for a receive happens before a following send if this order is enforced by a process (step 4); for a receive, the nearest-enclosing wait happens before the nearest-enclosing barrier (step 5); barriers happen before operations after it (step 6); and sends are received in the order they are sent (step 7). 

\paragraph*{Step 1} For each process, if there are sequential send
operations, say $\mathtt{S}$ and $\mathtt{S^\prime}$, from that task
to a common endpoint, $e_\mathtt{S} = e_\mathtt{S^\prime}$, then those
sends must follow program order: $\mathit{order}_\mathtt{S}$
$\prec_\mathtt{HB}$ $\mathit{order}_\mathtt{S^\prime}$.

\paragraph*{Step 2} For each process, if there are sequential receive
operations, say $\mathtt{R}$ and $\mathtt{R^\prime}$, in that task
on a common endpoint, $e_\mathtt{R} = e_\mathtt{R^\prime}$, then those
receives must follow program order: $\mathit{order}_\mathtt{R}$
$\prec_\mathtt{HB}$ $\mathit{order}_\mathtt{R^\prime}$.

\paragraph*{Step 3} For every receive \texttt{R} and its nearest
enclosing wait \texttt{W}, $\mathit{order}_\mathtt{R}$
$\prec_\mathtt{HB}$ $\mathit{order}_\mathtt{W}$.

\paragraph*{Step 4} For each process, if there is a sequential order over the nearest-enclosing wait for a receive operation and a send operation, say $\mathtt{W}$ and $\mathtt{S}$, then they must follow program order: $\mathit{order}_\mathtt{W}$
$\prec_\mathtt{HB}$ $\mathit{order}_\mathtt{S}$.

\paragraph*{Step 5} For any receive \texttt{R} that has a nearest-enclosing barrier \texttt{B} and a nearest-enclosing wait \texttt{W}, they must follow the program order:
$\mathit{order}_\mathtt{W}$ $\prec_\mathtt{HB}$ $\mathit{order}_\mathtt{B}$.

\paragraph*{Step 6} For any barrier $\mathtt{B}$ that has an operation $\mathtt{O}$ issued after it, they must follow the program order: $\mathit{order}_\mathtt{B}$
$\prec_\mathtt{HB}$ $\mathit{order}_\mathtt{O}$.

\paragraph*{Step 7} For any pair of sends $\mathtt{S}$ and
$\mathtt{S'}$ on common endpoints, $e_{\mathtt{S}}=e_{\mathtt{S'}}$,
such that
$\mathit{order}_\mathtt{S}\ \mathrm{\prec_\mathtt{HB}}\ \mathit{order}_\mathtt{S'}$,
then those sends must be received in the same order:
$M_{\mathtt{S}}\ \mathrm{\prec_{\mathtt{HB}}}\ M_{\mathtt{S'}}$.

As discussed earlier, the message communication may not be affected by barriers. Therefore, step $5$ only constrains the program order over the nearest-enclosing wait and the nearest-enclosing barrier for a receive. The order over this receive and the nearest-enclosing barrier is not constrained. For step $6$, a barrier has to happen before any operations after it. Step $7$ enforces the non-overtaking order over two sends to a common process. So two matching receives, recoded in the ``$\mathtt{M}$" values, have to be ordered in a proper way.

%: DO WE MISS W(R) <_HB S IN A SEQUENTIAL ORDER for infinite buffering?


%: WHAT IF A SEND IS ISSUED BEFORE A RECEIVE or A RECEIVE IS ISSUED BEFORE A SEND IN AN IDENTIAL PROCESS?
%:under zero buffer, each process model should be ordered strictly.



%: show encoding for figure 4

\subsection{Zero Buffering Encoding}
The zero buffer semantics behave differently from the infinite buffer semantics.  The zero buffer semantics enforces each message to be delivered substantially once it is sent out. To correctly encode zero buffer semantics, we need to refine the existing rules. The significant change is the new \emph{happens-before} relation, we call \textit{happens-before*}, that further constrains the partial order over a send and its matching receive based on equality relation. 

\begin{definition}[Happens-Before*]
The \emph{Happens-Before*} $(\mathtt{HB^*})$ relation, denoted as
$\mathit{order}_\mathtt{e1} \mathrm{\prec_\mathtt{HB*}} \mathit{order}_\mathtt{e2}$, is a partial order over two consecutive events, $\mathtt{e1}$ and $\mathtt{e2}$ respectively, that constrains the logic formula $\mathit{order}_\mathtt{e1} =  \mathit{order}_\mathtt{e2} - 1$.
\label{def:hb*}
\end{definition}

The $\mathtt{HB^*}$ relation constrains the consecutive order over two events. With the help of this relation, a match pair is extended as follows: 

\begin{definition}[Match Pair *] \label{def:match*}
A match pair, $\langle\mathtt{R}, \mathtt{S}\rangle$, for a receive
$\mathtt{R}$ and a send $\mathtt{S}$ corresponds to the constraints:
\begin{compactenum}
\item $M_{\mathtt{R}} = \mathit{order}_{\mathtt{S}}$
\item $M_{\mathtt{S}} = \mathit{order}_{\mathtt{R}}$
\item $e_{\mathtt{R}} = e_{\mathtt{S}}$
\item $src_\mathtt{R} = src_\mathtt{S}$
\item $\mathit{value}_{\mathtt{R}} = \mathit{value}_{\mathtt{S}}$ and
\item $\mathit{order}_{\mathtt{S}}\ \mathrm{\prec_\mathtt{HB*}}\ \mathit{order}_{\mathtt{R}}$
\end{compactenum}
\end{definition}

Intuitively, the consecutive order over each send and the matching receive is constrained. As a result, any resolved execution alternately orders a send followed by a receive. To further constrain the program order for zero buffer semantics, new rules are added: two sends are ordered on each process (step 1*); a send happens before a receive issued later on each process (step 8); and similarly, a send happens before a barrier issued later on each process (step 9).

\paragraph*{Step 1*} For each process, if there are sequential send
operations, say $\mathtt{S}$ and $\mathtt{S^\prime}$, then those
sends must follow program order: $\mathit{order}_\mathtt{S}$
$\prec_\mathtt{HB}$ $\mathit{order}_\mathtt{S^\prime}$.

\paragraph*{Step 8} For each process, if there is a sequential order over a send operation and a receive operation, say $\mathtt{S}$ and $\mathtt{R}$, then they must follow program order: 
$\mathit{order}_\mathtt{S}$
$\prec_\mathtt{HB}$ $\mathit{order}_\mathtt{R}$.

\paragraph*{Step 9} For each process, if there is a sequential order over a send operation and a barrier operation, say $\mathtt{S}$ and $\mathtt{B}$, then they must follow program order: 
$\mathit{order}_\mathtt{S}$
$\prec_\mathtt{HB}$ $\mathit{order}_\mathtt{B}$.

Step $1^*$ extends step $1$ as zero buffer semantics do not allow a new send to be issued before the pending send is completed on a common process. Step $8$ and step $9$ constrain for the same purpose.

\subsection{Correctness}

\begin{definition}
For all SMT problems, $s$, let $\mathcal{SOL}(s)$ be in $\sigma +
\mathbb{UNSAT}$, where $\sigma$ is a satisfying assignment of
variables to values.
\end{definition}

\begin{definition}[Semantics]
For all programs, $p$, and traces $t$, $\mathcal{SEM}(p, t)$ is either
$\mathbb{BAD}$ or $\mathbb{OK}$.
\end{definition}

\begin{theorem}[Soundness]
For all programs, $p$, and match pair sets, $m$,
$\mathcal{SOL}(\mathcal{SMT}(p, m)) = t \Rightarrow \mathcal{SEM}(p, t) =
\mathbb{BAD}$.
\end{theorem}


\begin{lemma} \label{lem:bogus}
Any match pair $\langle \mathtt{R}, \mathtt{S}\rangle$ used in a
satisfying assignment of an SMT encoding is a valid match pair and
reflects an actual possible MCAPI program execution.
\end{lemma}

\begin{theorem}[Completeness]
For all programs, $p$, and traces, $t$, $\mathcal{SEM}(p, t) =
\mathbb{BAD} \Rightarrow \exists m . \mathcal{SOL}(\mathcal{SMT}(p,
m)) = t$.
\end{theorem}

\begin{theorem}[Approximation]
Give two match pair sets $m$ and $m'$, $m \subseteq m' \Rightarrow \mathcal{SOL}(\mathcal{SMT}(p, m))
  \sqsubseteq \mathcal{SOL}(\mathcal{SMT}(p, m'))$, where
  $\mathbb{UNSAT} \sqsubseteq \sigma$.
\end{theorem}

%\begin{figure}[h1]
%\begin{center}
%\setlength{\tabcolsep}{2pt}
%\small \begin{tabular}[t]{l}
%P2 $\mathtt{S(to\ P0, "4", \&h5)}$ \\
%P2 $\mathtt{S: 4}$\\
%P0 $\mathtt{R(from\ P2, A, \&h1)}$ \\
%P0 $\mathtt{R: 4}$ \\
%P2 $\mathtt{S(to\ P1, "Go", \&h6)}$ \\
%P2 $\mathtt{S: Go}$ \\
%P1 $\mathtt{R(from\ P2, C, \&h3)}$ \\
%P1 $\mathtt{R: Go}$ \\
%P1 $\mathtt{S(to\ P0, "1", \&h4)}$ \\
%P1 $\mathtt{S: 1}$ \\
%P0 $\mathtt{R(from\ P1, B, \&h2)}$ \\
%P0 $\mathtt{R: 1}$ \\
%\end{tabular}
%\end{center}
%\caption{The legal history of the MPI program execution in \figref{fig:mpi}}
%\label{fig:history}
%\end{figure}
%
%\begin{figure}[h2]
%\begin{center}
%\setlength{\tabcolsep}{2pt}
%\small \begin{tabular}[t]{l}
%P0 $\mathtt{R(from\ P2, A, \&h1)}$ \\
%P1 $\mathtt{R(from\ P2, C, \&h3)}$ \\
%P2 $\mathtt{S(to\ P0, "4", \&h5)}$ \\
%P2 $\mathtt{S: 4}$\\
%P0 $\mathtt{R: 4}$ \\
%P2 $\mathtt{S(to\ P1, "Go", \&h6)}$ \\
%P2 $\mathtt{S: Go}$ \\
%P1 $\mathtt{R: Go}$ \\
%P1 $\mathtt{S(to\ P0, "1", \&h4)}$ \\
%P1 $\mathtt{S: 1}$ \\
%P0 $\mathtt{R(from\ P1, B, \&h2)}$ \\
%P0 $\mathtt{R: 1}$ \\
%\end{tabular}
%\end{center}
%\caption{The second legal history of the MPI program execution in \figref{fig:mpi}}
%\label{fig:history}
%\end{figure}

So far, we have proved the infinite buffer encoding is correct. Next, we want to prove the correctness of the zero buffer encoding. As shown earlier, the zero buffer encoding extends the rules for match pair and program order. An essential rule is the the use of $\mathtt{HB^*}$ relation that constrains a send and its matching receive in a consecutive order. Based on this rule, the encoding assumes that a send and a receive can only be ordered alternately in any resolved execution. This assumption is inspired by Concurrent CBMC that assumes a lock and its paired unlock can only be ordered alternately in any execution \cite{DBLP:conf/cav/RabinovitzG05}. Similarly, the encoding in this paper detects the assertion violation by only considering this alternate order in an execution. One may argue that such a encoding dose not cover all possible ways of message communication without considering arbitrary order over sends and receives. However, we claim this assumption is enough to provide a full message communication topology. To prove it, we state a theorem below.  We also define a few terms to support the proof of this theorem.

We use a legal history to represent a total order over events. In a legal history, an operation is split into two events: method invocation and method response. The invocation is a event that starts this operation. It contains a method name and method arguments with deterministic values. The response is a event that ends this operation. It contains the return value for this operation. The following definition relies on both events.

\begin{definition}[Legal History]
Under zero buffer semantics, a history of an MPI program is a sequence of send and receive operations over method invocations and method responses. It is legal if
\begin{compactenum}
\item the partial order over method invocations obeys the steps \textit{1*} and \textit{2} $to$ \textit{9}; and 
\item each send and its matching receive are ordered consecutively if this send is invoked first; or such an order is not enforced if the receive is invoked first.
\end{compactenum}
\end{definition}

Unlike a program execution that has all sorts of events ordered, a legal history only takes care of send invocation, send response, receive invocation and receive response because they are essential to the message communication. In other words, these events can be used to evaluate how each message may flow in a runtime system. For a feasible message flow, the legal history has to satisfy two conditions in the definition. The first condition has already been defined in the encoding. Interestingly, the second condition states two possible ways that a send and its matching receive can be ordered. Since the zero buffer semantics deliver a message by copying it directly into a receive buffer, it only affects the partial order when a send is first invoked. If a receive is invoked first, however, there is no need to constrain this order. 

%As an example, \figref{fig:history} shows a legal history for \figref{fig:mpi}. In this history, each event is paired with a process rank. Each send and its matching receive are order alternately so that each message flows across a receive once it is sent out. This history also reveals the message communications by showing the sequence of all receive responses. 

Two legal histories can be compared for equivalency based on \defref{def:er}. Intuitively, two legal histories should receive a common sequence of messages on each process. 

\begin{definition}[Equivalency Relation]\label{def:er}
Two legal histories, say $H$ and $L$, are equivalent, denoted as $H$ $\sim$ $L$, if and only if their projections to the receive responses on each single process $p$, $H | p$ and $L | p$ respectively, agree on the order and value of messages received.
\end{definition}

It is easy to prove that this relation is reflexive, symmetric and transitive. Therefore, it is enough to identify the equivalent classes among all legal histories. The equivalent legal histories have a common sequence of received messages on each process. The following theorem further states that a representative legal history exists for each equivalent class. 

\begin{theorem}
For zero buffer semantics, there exists a legal history in each equivalent class that orders sends and receives alternately.
\end{theorem}

\begin{proof}
prove 
\end{proof}



 


