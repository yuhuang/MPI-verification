\section{SMT Encoding}
The SMT encoding is generated from 1) an execution trace of an MPI program that includes a sequence of events, i.e., point-to-point operations, collective operations, control flow assumptions and property assertions; and 2) a set of possible match pairs for message communication. Intuitively, a match pair is a coupling of a send and a receive. Taking \figref{fig:mpi} as an example, each receive on process $0$ may be matched with a send from process $1$ or process $2$. The direct use of match pair is easy to reason about message communication. The encoding also uses \textit{happens-before} relations stemming from the program order for each process model and each concurrency relation. A modern SMT solver, such as Yice \cite{dutertre:CAV06} or Z3 \cite{demoura:tacas08}, is asked to resolve this encoding. A resolved satisfying schedule is guaranteed to be feasible with a violation of user provided assertions in the system runtime. If no satisfying schedule exists, the correctness of the program is also proved (no violation of the assertion) in all possible executions. 

According to the discussion earlier, MPI behavior can be affected by two runtime settings: infinite buffer semantics and zero buffer semantics. The infinite buffer semantics behave differently from the zero buffer semantics in how messages communicate. Therefore, the encoding should use different rules to support both settings. 

This section first discusses the algorithm to encode infinite buffer semantics, including the rules for point-to-point operations, collective operations and how they are constrained in program order. It then discusses how to adjust this encoding to zero buffer semantics by adding a few new rules.

\subsection{Partial Order}
The encoding needs to express the partial order imposed by the MPI semantics as SMT constraints. Such a partial order needs to bind a ``timestamp" variable (\defref{def:order}) to each event in the program. The pertinent ``timestamp" variables of two events are used to constrain their partial order as a \emph{happens-before} relation (\defref{def:happens-before}).

\begin{definition}[Timestamp]\label{def:order}
The timestamp of an event $\mathtt{e}$, denoted as $\mathit{time}_\mathtt{e}$, is constrained as an integer.
\end{definition}

\begin{definition}[Happens-Before]\label{def:happens-before}
The \emph{Happens-Before} $(\mathtt{HB})$ relation, denoted as
$\mathit{time}_\mathtt{e1} \mathrm{\prec_\mathtt{HB}} \mathit{time}_\mathtt{e2}$, is a strict partial order over two events, $\mathtt{e1}$ and $\mathtt{e2}$ respectively, that constrains the logic formula $\mathit{time}_\mathtt{e1} <  \mathit{time}_\mathtt{e2}$. 
\label{def:hb}
\end{definition}

In \defref{def:happens-before}, if $\mathtt{e1}$ must happen before $\mathtt{e2}$ in any execution, the encoding needs to add $\mathit{time}_\mathtt{e1} \mathrm{\prec_\mathtt{HB}} \mathit{time}_\mathtt{e2}$ as a constraint. The goal of an SMT solver is to evaluate all the ``timestamp" variables in such a way that satisfies all the $\mathtt{HB}$ relations. Based on the $\mathtt{HB}$ relation, the encoding further defines a few rules to constrain the program order. 

\subsection{Point-To-Point Communication}
%:Definitions
Before discussing those rules, it is necessary to define MPI point-to-point operations as SMT constraints. The MPI point-to-point communication consists of two basic  primitives: send and receive. The encoding defines send (\defref{def:snd}) and receive (\defref{def:rcv}) as a set of variables to constrain the concurrent behavior. Some variables, such as the endpoint information, are already evaluated from the existing trace. The other variables, such as the order information, need to be resolved by an SMT solver. 
 
\begin{definition}[Send] \label{def:snd}
A send operation $\mathtt{S}$, is a four-tuple of variables:
\begin{compactenum}
\item $M_\mathtt{S}$, the timestamp of the matching receive event;
\item $\mathit{time}_\mathtt{S}$, the timestamp of the send;
\item $e_\mathtt{S}$, the destination endpoint; 
\item $src_\mathtt{S}$, the source endpoint; and
\item $\mathit{value}_\mathtt{S}$, the transmitted value.
\end{compactenum}
\end{definition}

As shown in \defref{def:snd}, the encoding of a send has a few evaluated variables. Those variables include the source endpoint, the destination endpoint and the transmitted value. As for the free variables, the encoding records the event timestamps for itself and the matching receive as $\mathit{time}_\mathtt{S}$ and $M_\mathtt{S}$ respectively. In particular, $M_\mathtt{S}$ is used to bind a specific receive if a message from $\mathtt{S}$ flows across this receive. Such a send-receive match is only resolved dynamically by an SMT solver. As long as the value assigned to $M_\mathtt{S}$, $\mathtt{S}$ is not allowed to match with another receive (otherwise it is a false constraint).

\begin{definition}[Wait] \label{def:wait}
The occurrence of a wait operation, \texttt{W}, is captured by a
single variable, $\mathit{time}_\mathtt{W}$, that constrains when
the wait occurs.
\end{definition}

In MPI semantics, a non-blocking send is paired with a wait operation that confirms the message has been copied out of the send buffer. The wait for a send does not affect message communication mainly because it is not able to detect a matching receive when it returns. As such, the encoding simply ignores the wait for a send. In contrast,  a wait for a receive is able to confirm message communication. The encoding defines such a wait operation in \defref{def:wait}. The wait for a receive witnesses that the message is copied into the receive buffer. So a matching send must be detected when the wait returns. This behavior confirms two things: 1) a receive is issued before its wait; and 2) a  send is issued before the wait of a matching receive. The encoding constrains this behavior with a few rules discussed later. Further, MPI semantics allow a single wait to witness the completion of one or more receives due to the message non-overtaking property. Such a wait is the nearest-enclosing wait. 

\begin{definition}[Nearest-Enclosing Wait] \label{def:nw}
A nearest-enclosing wait is a wait that witnesses the completion of a receive by indicating that
the message is delivered and that all the previous receives in the
same task issued earlier are complete as well.
\end{definition}
%: an example to show nearest-encolsing wait

The encoding requires that every receive has a nearest-enclosing wait so a match pair decision can be made at this wait. Based on this requirement, a receive should include the nearest-enclosing wait in its variables (\defref{def:rcv}).

\begin{definition}[Receive] \label{def:rcv}
A receive operation $\mathtt{R}$ is modeled by a five-tuple of variables:
\begin{compactenum}
\item $M_\mathtt{R}$, the order of the matching send event;
\item $\mathit{time}_\mathtt{R}$, the timestamp of the receive;
\item $e_\mathtt{R}$, the destination endpoint;
\item $src_\mathtt{R}$, the source endpoint;
\item $\mathit{value}_\mathtt{R}$, the received value; and,
\item $\mathit{nw}_\mathtt{R}$, the timestamp of the nearest enclosing wait.
\end{compactenum}
\end{definition}

The encoding of a receive has a variable for the destination endpoint that is evaluated from the existing trace. It defines two free variables $\mathit{time}_\mathtt{R}$ and $M_\mathtt{R}$, for the event timestamps of itself and the matching send, respectively. The transmitted (received) value is not known until a send is matched. Interestingly, the source endpoint for a receive may or may not be known depending which receive it defines. MPI semantics support both deterministic receive with a constant source endpoint and wildcard receive with a uncertain source endpoint. The nearest-enclosing wait, as discussed earlier, is used to make a match pair decision. 

The encoding uses match pairs directly to express the message communication. Intuitively, \defref{def:match} asserts that $\mathtt{R}$ and $\mathtt{S}$ are matched with identical event timestamps, endpoints and transmitted values. Also, $\mathtt{S}$ should be ordered before the nearest-enclosing wait of $\mathtt{R}$. 

\begin{definition}[Match Pair] \label{def:match}
A match pair, $\langle\mathtt{R}, \mathtt{S}\rangle$, for a receive
$\mathtt{R}$ and a send $\mathtt{S}$ corresponds to the constraints:
\begin{compactenum}
\item $M_{\mathtt{R}} = \mathit{time}_{\mathtt{S}}$
\item $M_{\mathtt{S}} = \mathit{time}_{\mathtt{R}}$
\item $e_{\mathtt{R}} = e_{\mathtt{S}}$
\item $src_\mathtt{R} = src_\mathtt{S}$
\item $\mathit{value}_{\mathtt{R}} = \mathit{value}_{\mathtt{S}}$ and
\item $\mathit{time}_{\mathtt{S}}\ \mathrm{\prec_\mathtt{HB}}\ \mathit{nw}_{\mathtt{R}}$
\end{compactenum}
\end{definition}

A set of match pairs correspond to a message communication topology. There may exist several match pairs for a single receive indicating more messages may flow across this receive. However, there is no way to receive all the messages in a single execution, even though all the match pairs should be considered. Therefore, the encoding does not combine these match pairs in a single conjunction. Instead, it constrains them for a single receive in a disjunction (\defref{def:receive_match}). As the $M$ values for both a receive and a send are deterministic once they are matched, only one match pair can be used in a final resolution.

\begin{definition}[Receive Matches] \label{def:receive_match}
For each receive $\mathtt{R}$, if $\langle\mathtt{R},
\mathtt{S}_0\rangle$ through $\langle\mathtt{R}, \mathtt{S}_n\rangle$
are match pairs, then $\bigvee_{i}^{n} \langle\mathtt{R},
\mathtt{S}_i\rangle$ is used as an SMT constraint.
\end{definition}

\subsection{Collective Communication}
Collective communication is also significant in MPI semantics. There are various collective operations used for this purpose.  Taking barrier as an example, it is used as a group meaning each process should issue a barrier.  It synchronizes the program in such a way that each process waits at some location until all group members are completed. As a result, an operation issued before a barrier cannot interleave any operation after the barrier. Other collective operations are used in a common way except that they additionally address a few tasks such as internal message communication and computation. Those tasks are not interrupted by MPI point-to-point communication, and vice versa. According to this fact, the algorithm only needs to consider how to constrain the behavior of synchronization, and simply constrains the additional tasks as assertions. We take barrier in the following discussion as it is merely used for program synchronization. The encoding defines the barrier in \defref{def:barrier}. 

\begin{definition}[Barrier]\label{def:barrier}
The occurrence of a barrier operation, \texttt{B}, is captured by a
single variable, $\mathit{time}_\mathtt{B}$, that constrains when a group of barriers $\{B_0, B_1, ..., B_n\}$ are matched.  
Each barrier $B_i, i\in{0 ... n}$, is issued by process $i$. 
\end{definition}

\begin{figure}[h]
\[
\begin{array}{l|l}
\;\;\;\;\;\;\;\;\mathtt{Process\ 0}\;\;\;\;\;\;\;\; & \;\;\;\;\;\;\;\; \mathtt{Process\ 1}\;\;\;\;\;\;\;\; \\
\hline
\;\;\;\;\;\;\;\;\mathtt{\underline{B(comm)}}\;\;\;\;\;\;\;\; & \;\;\;\;\;\;\;\; \mathtt{R(from\ P0,A\&h2)}\;\;\;\;\;\;\;\; \\
\;\;\;\;\;\;\;\;\mathtt{S(to\ P1,``1",\&h1)}\;\;\;\;\;\;\;\; & \;\;\;\;\;\;\;\; \mathtt{\underline{B(comm)}}\;\;\;\;\;\;\;\; \\
\;\;\;\;\;\;\;\;\mathtt{W(\&h1)}\;\;\;\;\;\;\;\; & \;\;\;\;\;\;\;\; \mathtt{W(\&h2)}\;\;\;\;\;\;\;\; \\
\end{array}
\]
\caption{An Example of Message Communication with Barriers} \label{fig:mc_barrier1}
\end{figure}

Even though barriers affect the issuing order over two events, it is hard to determine whether they prevent a send from matching a receive. As an example, the message ``$1$" in \figref{fig:mc_barrier1} may flow across $\mathtt{R}$ even though $\mathtt{R}$ is ordered before the barriers and $\mathtt{S}$ is ordered after the barriers. However, if the program had issued $\mathtt{W(\&h2)}$ before the barriers, $\mathtt{R}$ would have to be completed before the barriers, meaning a message had to be delivered. In this situation, message communication is affected. The algorithm further defines the \textit{nearest-enclosing barrier} (\defref{def:nb}) to constrain this affection.

\begin{definition}[Nearest-Enclosing Barrier]\label{def:nb}
For a process $i$, a receive \texttt{R} has a nearest-enclosing barrier \texttt{B} if and only if
\begin{compactenum}
\item the nearest-enclosing wait \texttt{W} of \texttt{R} is ordered before $B_i\in B$, and
\item there does not exist any receive \texttt{R'} such that its nearest-enclosing wait \texttt{W'} is ordered after \texttt{W} and is ordered before $B_i$.
\end{compactenum}
\end{definition}


\subsection{Assumptions and Assertions}
Except MPI point-to-point operations and MPI collective operations, assumptions and assertions should be constrained as well. An assumption asks any feasible execution to obey the imposed control flow path, so it can be viewed as a inviolated truth in any execution. The encoding constrains this truth as an SMT assertion.
\begin{definition}[Assumption] \label{def:assm}
Every assumption $\mathtt{A}$ is added as an SMT assertion.
\end{definition}

An assertion checks if a feasible execution holds the property that is interesting to the user. The goal of our encoding is to detect a hidden assertion violation in any execution. Therefore, the \textit{negation} of this property should be encoded so any satisfying assignment corresponds to a witness of violation. 

\begin{definition}[Property Assertion] \label{def:assert}
For every property assertion $\mathtt{P}$, $\neg \mathtt{P}$ is added as
an SMT assertion.
\end{definition}

\subsection{Program Order}
The encoding thus far defines a few terms that are used to constrain the program order for infinite buffer semantics. The program order can be added in seven steps: we must ensure that two sends to a common endpoint must be ordered on each process (step 1); similar to the receives (step 2); receives happen before their nearest-enclosing wait (step 3); the nearest-enclosing wait for a receive happens before a following send if this order is enforced by a process (step 4); for a receive, the nearest-enclosing wait happens before the nearest-enclosing barrier (step 5); a barrier happens before any operation after it (step 6); and sends are received in the order they are sent (step 7). 

\paragraph*{Step 1} For each process, if there are sequential send
operations, say $\mathtt{S}$ and $\mathtt{S^\prime}$, from that task
to a common endpoint, $e_\mathtt{S} = e_\mathtt{S^\prime}$, then those
sends must follow program order: $\mathit{time}_\mathtt{S}$
$\prec_\mathtt{HB}$ $\mathit{time}_\mathtt{S^\prime}$.

\paragraph*{Step 2} For each process, if there are sequential receive
operations, say $\mathtt{R}$ and $\mathtt{R^\prime}$, in that task
on a common endpoint, $e_\mathtt{R} = e_\mathtt{R^\prime}$, then those
receives must follow program order: $\mathit{time}_\mathtt{R}$
$\prec_\mathtt{HB}$ $\mathit{time}_\mathtt{R^\prime}$.

\paragraph*{Step 3} For every receive \texttt{R} and its nearest
enclosing wait \texttt{W}, $\mathit{time}_\mathtt{R}$
$\prec_\mathtt{HB}$ $\mathit{time}_\mathtt{W}$.

\paragraph*{Step 4} For each process, if there is a sequential order over the nearest-enclosing wait for a receive operation and a send operation, say $\mathtt{W}$ and $\mathtt{S}$, then they must follow program order: $\mathit{time}_\mathtt{W}$
$\prec_\mathtt{HB}$ $\mathit{time}_\mathtt{S}$.

\paragraph*{Step 5} For any receive \texttt{R} that has a nearest-enclosing barrier \texttt{B} and a nearest-enclosing wait \texttt{W}, they must follow the program order:
$\mathit{time}_\mathtt{W}$ $\prec_\mathtt{HB}$ $\mathit{time}_\mathtt{B}$.

\paragraph*{Step 6} For any barrier $\mathtt{B}$ and an operation $\mathtt{O}$ ordered after $\mathtt{B}$, they must follow the program order: $\mathit{time}_\mathtt{B}$
$\prec_\mathtt{HB}$ $\mathit{time}_\mathtt{O}$.

\paragraph*{Step 7} For any pair of sends $\mathtt{S}$ and
$\mathtt{S'}$ on common endpoints, $e_{\mathtt{S}}=e_{\mathtt{S'}}$,
such that
$\mathit{time}_\mathtt{S}\ \mathrm{\prec_\mathtt{HB}}\ \mathit{time}_\mathtt{S'}$,
then those sends must be received in the same order:
$M_{\mathtt{S}}\ \mathrm{\prec_{\mathtt{HB}}}\ M_{\mathtt{S'}}$.

As discussed earlier, message communication may not be affected by barriers. Therefore, step $5$ only constrains the program order over the nearest-enclosing wait and the nearest-enclosing barrier for a receive. The order over this receive and the nearest-enclosing barrier is not constrained. For step $6$, a barrier has to happen before any operation ordered after it. Step $7$ enforces the non-overtaking order over two sends to a common process. So two matching receives, recoded in the ``$\mathtt{M}$" values, have to be ordered in a proper way.

%: DO WE MISS W(R) <_HB S IN A SEQUENTIAL ORDER for infinite buffering?


%: WHAT IF A SEND IS ISSUED BEFORE A RECEIVE or A RECEIVE IS ISSUED BEFORE A SEND IN AN IDENTIAL PROCESS?
%:under zero buffer, each process model should be ordered strictly.



%: show encoding for figure 4

\subsection{Zero Buffer Encoding}
The zero buffer semantics behave differently from the infinite buffer semantics.  The zero buffer semantics enforce messages to be delivered substantially once they are sent out. To correctly encode zero buffer semantics, we need to refine the existing rules. The significant change is the new \emph{happens-before} relation, we call \textit{happens-before*}, that further constrains the partial order over a send and the matching receive based on equality relation. 

\begin{definition}[Happens-Before*]
The \emph{Happens-Before*} $(\mathtt{HB^*})$ relation, denoted as
$\mathit{time}_\mathtt{e1} \mathrm{\prec_\mathtt{HB*}} \mathit{time}_\mathtt{e2}$, is a successor relation over two consecutive events, $\mathtt{e1}$ and $\mathtt{e2}$ respectively, that constrains the logic formula $\mathit{time}_\mathtt{e1} =  \mathit{time}_\mathtt{e2} - 1$.
\label{def:hb*}
\end{definition}

The $\mathtt{HB^*}$ relation constrains the consecutive order over two events. With the help of this relation, a match pair is extended as follows: 

\begin{definition}[Match Pair *] \label{def:match*}
A match pair, $\langle\mathtt{R}, \mathtt{S}\rangle$, for a receive
$\mathtt{R}$ and a send $\mathtt{S}$ corresponds to the constraints:
\begin{compactenum}
\item $M_{\mathtt{R}} = \mathit{time}_{\mathtt{S}}$
\item $M_{\mathtt{S}} = \mathit{time}_{\mathtt{R}}$
\item $e_{\mathtt{R}} = e_{\mathtt{S}}$
\item $src_\mathtt{R} = src_\mathtt{S}$
\item $\mathit{value}_{\mathtt{R}} = \mathit{value}_{\mathtt{S}}$ and
\item $\mathit{time}_{\mathtt{S}}\ \mathrm{\prec_\mathtt{HB*}}\ \mathit{time}_{\mathtt{R}}$
\end{compactenum}
\end{definition}

Intuitively, the consecutive order over a send and the matching receive is constrained. As a result, any resolved execution alternatingly orders a send and a receive in a sequence. To further constrain the program order for zero buffer semantics, new rules are added: two sends are ordered on each process (step 1*); on each process a send happens before a receive that is issued later (step 8); and similarly, on each process a send happens before a barrier that is issued later (step 9).

\paragraph*{Step 1*} For each process, if there are sequential send
operations, say $\mathtt{S}$ and $\mathtt{S^\prime}$, then those
sends must follow program order: $\mathit{time}_\mathtt{S}$
$\prec_\mathtt{HB}$ $\mathit{time}_\mathtt{S^\prime}$.

\paragraph*{Step 8} For each process, if there is a sequential order over a send operation and a receive operation, say $\mathtt{S}$ and $\mathtt{R}$, then they must follow program order: 
$\mathit{time}_\mathtt{S}$
$\prec_\mathtt{HB}$ $\mathit{time}_\mathtt{R}$.

\paragraph*{Step 9} For each process, if there is a sequential order over a send operation and a barrier operation, say $\mathtt{S}$ and $\mathtt{B}$, then they must follow program order: 
$\mathit{time}_\mathtt{S}$
$\prec_\mathtt{HB}$ $\mathit{time}_\mathtt{B}$.

Step $1^*$ extends step $1$ as zero buffer semantics do not allow a new send to be issued before the pending send is completed on a common process. Step $8$ and step $9$ constrain the sequential order over a send and a receive and the sequential order over a send and a barrier, respectively.

\subsection{Correctness}

The prior work \cite{DBLP:conf/kbse/HuangMM13} has proved our encoding technique is correct over point-to-point communication for message passing. In particular, the encoding is sound, meaning any satisfying schedule resolved by the encoding reflects an actual violated execution. It is also complete, meaning any actual violation can be resolved in a satisfying schedule by the encoding. To prove our new technique with collective communication is also correct, we rely on the existing soundness and completeness proofs. For soundness, it is consistent with the prior proof and relies on the following lemma: 

\begin{lemma} \label{lem:bogus}
Any match pair $\langle \mathtt{R}, \mathtt{S}\rangle$ used in a
satisfying assignment of an SMT encoding is a valid match pair and
reflects an actual possible MPI program execution.
\end{lemma}

The completeness proof of our new technique extends the operational semantics in the prior work to support collective operations. It simulates the solving of the SMT problem during its operation in the extended semantics to ensure that the two make identical conclusions. Please refer to \cite{DBLP:conf/kbse/HuangMM13} for the complete proofs.

%\begin{figure}[h1]
%\begin{center}
%\setlength{\tabcolsep}{2pt}
%\small \begin{tabular}[t]{l}
%P2 $\mathtt{S(to\ P0, "4", \&h5)}$ \\
%P2 $\mathtt{S: 4}$\\
%P0 $\mathtt{R(from\ P2, A, \&h1)}$ \\
%P0 $\mathtt{R: 4}$ \\
%P2 $\mathtt{S(to\ P1, "Go", \&h6)}$ \\
%P2 $\mathtt{S: Go}$ \\
%P1 $\mathtt{R(from\ P2, C, \&h3)}$ \\
%P1 $\mathtt{R: Go}$ \\
%P1 $\mathtt{S(to\ P0, "1", \&h4)}$ \\
%P1 $\mathtt{S: 1}$ \\
%P0 $\mathtt{R(from\ P1, B, \&h2)}$ \\
%P0 $\mathtt{R: 1}$ \\
%\end{tabular}
%\end{center}
%\caption{The legal history of the MPI program execution in \figref{fig:mpi}}
%\label{fig:history}
%\end{figure}
%
%\begin{figure}[h2]
%\begin{center}
%\setlength{\tabcolsep}{2pt}
%\small \begin{tabular}[t]{l}
%P0 $\mathtt{R(from\ P2, A, \&h1)}$ \\
%P1 $\mathtt{R(from\ P2, C, \&h3)}$ \\
%P2 $\mathtt{S(to\ P0, "4", \&h5)}$ \\
%P2 $\mathtt{S: 4}$\\
%P0 $\mathtt{R: 4}$ \\
%P2 $\mathtt{S(to\ P1, "Go", \&h6)}$ \\
%P2 $\mathtt{S: Go}$ \\
%P1 $\mathtt{R: Go}$ \\
%P1 $\mathtt{S(to\ P0, "1", \&h4)}$ \\
%P1 $\mathtt{S: 1}$ \\
%P0 $\mathtt{R(from\ P1, B, \&h2)}$ \\
%P0 $\mathtt{R: 1}$ \\
%\end{tabular}
%\end{center}
%\caption{The second legal history of the MPI program execution in \figref{fig:mpi}}
%\label{fig:history}
%\end{figure}

In addition to the soundness and completeness, the improved zero buffer encoding needs to prove the coverage of message communication. As shown earlier, the zero buffer encoding extends the rules for match pair and program order. An essential rule is the the use of $\mathtt{HB^*}$ relation that constrains the consecutive order over a send and the matching receive. Based on this rule, the encoding assumes that a send and a receive is only ordered alternatingly in any resolved execution. This assumption is inspired by TCBMC that assumes each lock operation and its paired unlock operation is ordered alternately in any execution \cite{DBLP:conf/cav/RabinovitzG05}. Similarly, the encoding in this paper detects assertion violations by only considering this alternate order in an execution. One may argue that such a encoding does not cover all possible ways of message communication without considering arbitrary order over sends and receives. However, we claim this assumption is enough to capture a full message communication topology. To prove it, we state a theorem later.  Before that, we define a few terms to support the proof.

\begin{definition}[Legal History]\label{def:legal}
Under zero buffer semantics, a history of an MPI program is a sequence of send and receive operations with return values. It is legal if
\begin{compactenum}
\item the partial order over events obeys step \textit{1*} and step \textit{2} through step \textit{9}; and 
\item the matching receive is consecutively ordered before a send; or the matching receive is issued earlier than the send.
\end{compactenum}
\end{definition}

We use a legal history to represent a total order over events for an MPI program (\defref{def:legal}). A legal history only takes care of sends and receives because they are essential to message communication. In other words, the events in a legal history can be used to evaluate how each message may flow in a runtime system. Since the return value is deterministic for each operation, a legal history corresponds to a precise resolution of message communication. For a feasible message delivery, the legal history has to satisfy two conditions as shown in \defref{def:legal}. The first condition is to enforce the partial order over events that is already defined in the encoding. The second condition states two possible ways a send and the matching receive is ordered. Since zero buffer semantics do not advance the execution on a process if a pending message is not delivered, the legal history only needs to ensure that the matching receive is consecutively ordered after the send or the matching receive is issued earlier than the send. 

%As an example, \figref{fig:history} shows a legal history for \figref{fig:mpi}. In this history, each event is paired with a process rank. Each send and its matching receive are order alternately so that each message flows across a receive once it is sent out. This history also reveals the message communications by showing the sequence of all receive responses. 

Two legal histories can be compared for equivalency based on \defref{def:er}.  

\begin{definition}[Equivalency Relation]\label{def:er}
Two legal histories for an identical MPI program, say $H$ and $L$ respectively, are equivalent, denoted as $H$ $\sim$ $L$, if and only if their projections to all the receives on each single process $p$, $H | p$ and $L | p$ respectively, agree on the received messages with identical sequence and values.
\end{definition}

The equivalency relation is reflexive, symmetric and transitive. Therefore, it can be used to identify the equivalent classes among all legal histories for a given MPI program. A equivalent class of legal histories asserts two aspects: first, any two legal histories in this class are equivalent based on received message sequences; second, a legal history outside this class is not equivalent to any legal history in this class. The following theorem further states that a representative exists for each zero-buffer equivalent class of legal histories. 

\begin{theorem}
For any MPI program, each equivalent class of legal histories has a representative that orders sends and receives alternatingly.
\end{theorem}

\begin{proof}
First, assume there is a legal history $\mathtt{H}$ for a set of sends $\mathit{S}$ and a set of receives $\mathit{R}$. 
Projecting $\mathtt{H}$ to $\mathit{R}$ produces a message sequence. This sequence reflects how messages are received in $\mathtt{H}$ for all processes. Since the message communication is deterministic so each receive in $\mathit{R}$ is matched with a send in $\mathit{S}$. Inserting the matching send preceding each receive in the projection produces another sequence that alternatingly orders sends and receives. This sequence obeys two conditions stated in \defref{def:legal}, therefore, is a legal history. Further, it is equivalent to $\mathtt{H}$ as they receive a common sequence of messages on each process.
Therefore, for any existing legal history, the procedure above is able to find a equivalent legal history that alternatingly orders sends and receives.  
\end{proof}



 


