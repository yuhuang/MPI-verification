\section{SMT Encoding}
The SMT encoding is generated from 1) an execution trace of an MPI program that includes a sequence of events (i.e., point-to-point operations, collective operations, control flow assumptions and property assertions); and 2) a set of possible match pairs for message communication. Intuitively, a match pair is a coupling of a send and a receive. Taking \figref{fig:mpi} as an example, each receive on process $0$ may be matched with a send from process $1$ or process $2$. The direct use of match pairs simplifies reasoning about message communication. 

\subsection{Point-To-Point Communication}
The prior work has already defined a few terms that are still helpful to encode the point-to-point communication in this paper. Those definitions include timestamp (\defref{def:order}), Happens-Before relation (\defref{def:happens-before}), send (\defref{def:snd}), receive (\defref{def:rcv}), wait (\defref{def:wait}), nearest-enclosing wait (\defref{def:nw}), match pair (\defref{def:match}) and potential sends (\defref{def:potential}).

\begin{definition}[Timestamp]\label{def:order}
\noindent The timestamp of an event, $\mathtt{e}$, denoted as $\mathit{time}_\mathtt{e}$, is an integer.
\end{definition}

\begin{definition}[Happens-Before]\label{def:happens-before}
The \emph{Happens-Before} $(\mathtt{HB})$ operator, denoted as
$\mathrm{\prec_\mathtt{HB}}$, defines a partial order over two events, $\mathtt{e1}$ and $\mathtt{e2}$ respectively, such that $\mathit{time}_\mathtt{e1} <  \mathit{time}_\mathtt{e2}$. 
\end{definition}

\begin{definition}[Send] \label{def:snd}
A send operation $\mathtt{S}$ $=$ $(M_\mathtt{S},$ $\mathit{time}_\mathtt{S},$ $e_\mathtt{S},$ $\mathit{value}_\mathtt{S})$, is a four-tuple of variables:
\begin{compactenum}
\item $M_\mathtt{S}$, the timestamp of the matching receive event;
\item $\mathit{time}_\mathtt{S}$, the timestamp of the send;
\item $e_\mathtt{S}$, the destination endpoint; and
\item $\mathit{value}_\mathtt{S}$, the transmitted value.
\end{compactenum}
\end{definition}

\begin{definition}[Receive] \label{def:rcv}
A receive operation $\mathtt{R}$ $=$ $(M_\mathtt{R},$ $\mathit{time}_\mathtt{R},$ $e_\mathtt{R},$ $\mathit{value}_\mathtt{R},$ $\mathit{time}_{\mathit{w}_\mathtt{R}})$, is modeled by a five-tuple of variables:
\begin{compactenum}
\item $M_\mathtt{R}$, the timestamp of the matching send event;
\item $\mathit{time}_\mathtt{R}$, the timestamp of the receive;
\item $e_\mathtt{R}$, the destination endpoint;
\item $\mathit{value}_\mathtt{R}$, the received value; and,
\item $\mathit{time}_{\mathit{w}_\mathtt{R}}$, the timestamp of the nearest enclosing wait.
\end{compactenum}
\end{definition}

\begin{definition}[Wait] \label{def:wait}
The occurrence of a wait operation, \texttt{W}, is captured by a
single variable, $\mathit{time}_\mathtt{W}$, that constrains when
the wait occurs.
\end{definition}

\begin{definition}[Nearest-Enclosing Wait] \label{def:nw}
A nearest-enclosing wait is a wait that witnesses the completion of a receive by indicating that
the message is delivered and that all the previous receives in the
same task issued earlier are complete as well.
\end{definition}

\begin{definition}[Match Pair] \label{def:match}
A match pair, $\langle\mathtt{R}, \mathtt{S}\rangle$, for a receive 
$\mathtt{R}$ $=$ $(M_\mathtt{R},$ $\mathit{time}_\mathtt{R},$ $e_\mathtt{R},$ $\mathit{value}_\mathtt{R},$ $\mathit{time}_{\mathit{w}_\mathtt{R}})$ and a send $\mathtt{S}$ $=$ $(M_\mathtt{S},$ $\mathit{time}_\mathtt{S},$ $e_\mathtt{S},$ $\mathit{value}_\mathtt{S})$, corresponds to the following constraints:
\begin{compactenum}
\item $M_{\mathtt{R}} = \mathit{time}_{\mathtt{S}}$
\item $M_{\mathtt{S}} = \mathit{time}_{\mathtt{R}}$
\item $e_{\mathtt{R}} = e_{\mathtt{S}}$
\item $\mathit{value}_{\mathtt{R}} = \mathit{value}_{\mathtt{S}}$ and
\item $\mathit{time}_{\mathtt{S}}\ \mathrm{\prec_\mathtt{HB}}\ \mathit{time}_{\mathit{w}_\mathtt{R}}$
\end{compactenum}
\end{definition}

\begin{definition}[Potential Sends]\label{def:potential}
The potential sends for a receive $\mathtt{R}$, denoted as $\mathrm{Match}(\mathtt{R})$, is a set of all the sends that $\mathtt{R}$ may potentially match.
\end{definition}

Based on the definitions above, the prior work encodes the point-to-point communication in four steps: 
\begin{compactenum}
\item Every assumption $\mathtt{A}$ on control flow is added as an SMT assertion;
\item For every property assertion $\mathtt{P}$, $\neg \mathtt{P}$ is added as an SMT assertion;
\item For each receive $\mathtt{R}$, $\bigvee_{\mathtt{S}_i \in \mathrm{Match}(\mathtt{R})} \langle\mathtt{R},\mathtt{S}_i\rangle$ is add as an SMT constraint; and
\item The program order over two events are add as SMT constraints. 
\end{compactenum}

The program order in step $4$ is further defined in a set of rules: 

\paragraph*{Rule 1} For each process, if there are sequential send
operations, say $\mathtt{S}$ and $\mathtt{S^\prime}$, from that process 
to a common endpoint, $e_\mathtt{S} = e_\mathtt{S^\prime}$, then those
sends must follow program order: $\mathit{time}_\mathtt{S}$
$\prec_\mathtt{HB}$ $\mathit{time}_\mathtt{S^\prime}$.

\paragraph*{Rule 2} For each process, if there are sequential receive
operations, say $\mathtt{R}$ and $\mathtt{R^\prime}$, such that $e_\mathtt{R} = e_\mathtt{R^\prime}$, then those
receives must follow program order: $\mathit{time}_\mathtt{R}$
$\prec_\mathtt{HB}$ $\mathit{time}_\mathtt{R^\prime}$.

\paragraph*{Rule 3} For every receive \texttt{R} and its nearest-enclosing wait \texttt{W}, $\mathit{time}_\mathtt{R}$ $\prec_\mathtt{HB}$ $\mathit{time}_\mathtt{W}$.

\paragraph*{Rule 4} For each process, if there is a sequential order over the nearest-enclosing wait for a receive operation and a send operation, say $\mathtt{W}$ and $\mathtt{S}$, then they must follow program order: $\mathit{time}_\mathtt{W}$
$\prec_\mathtt{HB}$ $\mathit{time}_\mathtt{S}$.

\paragraph*{Rule 5} For any pair of sends, $\mathtt{S}$ $=$ $(M_\mathtt{S},$ $\mathit{time}_\mathtt{S},$ $e_\mathtt{S},$ $\mathit{value}_\mathtt{S})$ and
$\mathtt{S^\prime}$ $=$ $(M_\mathtt{S^\prime},$ $\mathit{time}_\mathtt{S^\prime},$ $e_\mathtt{S^\prime},$ $\mathit{value}_\mathtt{S^\prime})$, on common endpoints, $e_{\mathtt{S}}=e_{\mathtt{S^\prime}}$,
such that
$\mathit{time}_\mathtt{S}\ \mathrm{\prec_\mathtt{HB}}\ \mathit{time}_\mathtt{S^\prime}$,
then those sends must be received in the same order:
$M_{\mathtt{S}}\ \mathrm{\prec_{\mathtt{HB}}}\ M_{\mathtt{S^\prime}}$.

\subsection{Deterministic Communication}
As the prior work does not encode deterministic point-to-point communication that is allowed in MPI semantics, we need to support it in the new encoding. The step is trivial. To be precise, the send operation defined in \defref{def:snd} and the receive operation defined in \defref{def:rcv} should add variables $src_\mathtt{S}$ and $src_\mathtt{R}$ respectively, for source endpoints. In addition, the match pair defined in \defref{def:match} should add a new constraint: 
\begin{equation*}
src_\mathtt{S} = src_\mathtt{R}, 
\end{equation*}
indicating that the source endpoints are matched for $\mathtt{S}$ and $\mathtt{R}$.

\subsection{Collective Communication}
As discussed earlier, the use of any type of collective operation is able to synchronize the program. In addition, some type of collective operation such as broadcast is able to send internal messages among its group, and/or to execute global operations. MPI semantics guarantee that messages generated on behalf of collective operations are not confused with messages generated by point-to-point operations. Therefore, the encoding in this paper puts emphasis on how to reason about the synchronization of collective operations as it may affect message communication. The internal message communication and the execution of global operations for collective communication can be simply added as SMT constraints to the encoding. In the following discussion, we take barrier as an example to encode the synchronization behavior. It is defined as a group in \defref{def:barrier}. 

\begin{definition}[Barrier]\label{def:barrier}
The occurrence of a group of barriers, $B$ = $\{B_0, B_1, ..., B_n\}$, is captured by a
single timestamp, $\mathit{time}_\mathtt{B}$, that marks when all the members in the group are matched.  
\end{definition}

\begin{figure}[h]
\[
\begin{array}{l|l}
\;\;\;\;\;\;\;\;\mathtt{Process\ 0}\;\;\;\;\;\;\;\; & \;\;\;\;\;\;\;\; \mathtt{Process\ 1}\;\;\;\;\;\;\;\; \\
\hline
\;\;\;\;\;\;\;\;\mathtt{\underline{B(comm)}}\;\;\;\;\;\;\;\; & \;\;\;\;\;\;\;\; \mathtt{R(from\ P0,A\&h2)}\;\;\;\;\;\;\;\; \\
\;\;\;\;\;\;\;\;\mathtt{S(to\ P1,``1",\&h1)}\;\;\;\;\;\;\;\; & \;\;\;\;\;\;\;\; \mathtt{\underline{B(comm)}}\;\;\;\;\;\;\;\; \\
\;\;\;\;\;\;\;\;\mathtt{W(\&h1)}\;\;\;\;\;\;\;\; & \;\;\;\;\;\;\;\; \mathtt{W(\&h2)}\;\;\;\;\;\;\;\; \\
\end{array}
\]
\caption{An Example of Message Communication with Barriers} \label{fig:mc_barrier1}
\end{figure}

Even though barriers affect the issuing order of two events, it is hard to determine whether they prevent a send from matching a receive. As an example, the message ``$1$" in \figref{fig:mc_barrier1} may flow into $\mathtt{R}$ even though $\mathtt{R}$ is ordered before the barrier and $\mathtt{S}$ is ordered after the barrier. The wait $\mathtt{W(\&h2)}$ determines the behavior. If the program had issued $\mathtt{W(\&h2)}$ before the barrier, $\mathtt{R}$ would have to be completed before the barrier, meaning the message is delivered. The algorithm further defines \textit{nearest-enclosing barrier} (\defref{def:nb}) for this type of interaction.

\begin{definition}[Nearest-Enclosing Barrier]\label{def:nb}
For any process $i$, a receive $\mathtt{R}$ has a nearest-enclosing barrier $\mathtt{B}$ if and only if
\begin{compactenum}
\item the nearest-enclosing wait, $\mathtt{W}$, of $\mathtt{R}$ is ordered before $B_i\in B$, and
\item there does not exist any receive $\mathtt{R^\prime}$ on process $i$, with a nearest-enclosing wait, $\mathtt{W^\prime}$, such that 1) $\mathtt{W^\prime}$ is ordered after $\mathtt{W}$; and 2) $\mathtt{W^\prime}$ is ordered before $B_i$.
\end{compactenum}
\end{definition}

Based on the definitions above, The algorithm defines two rules for program order:

\paragraph*{Rule 6} For any receive \texttt{R} that has a nearest-enclosing barrier \texttt{B} and a nearest-enclosing wait \texttt{W}, they must follow the program order:
$\mathit{time}_\mathtt{W}$ $\prec_\mathtt{HB}$ $\mathit{time}_\mathtt{B}$.

\paragraph*{Rule 7} For any barrier $\mathtt{B}$ and any operation $\mathtt{O}$ ordered after a member of $\mathtt{B}$, they must follow the program order: $\mathit{time}_\mathtt{B}$
$\prec_\mathtt{HB}$ $\mathit{time}_\mathtt{O}$.

Rule $6$ only constrains the program order over the nearest-enclosing wait and the nearest-enclosing barrier for a receive. The order over this receive and the nearest-enclosing barrier is not constrained. For rule $7$, a barrier has to happen before any operation ordered after it. 

\subsection{Zero Buffer Encoding}
The zero buffer semantics enforce messages to be delivered once they are sent. The prior work \cite{DBLP:conf/kbse/HuangMM13} provides a complicated method that fails to encode the valid behavior of zero buffer semantics. This paper presents a new zero buffer encoding technique that is correct and is easy to reason about zero buffer semantics. To be precise, we need to refine the existing rules for the infinite buffer encoding. The significant change is to order a send immediately preceding the matching receive in a match pair. Therefore, the match pair is defined in \defref{def:match*}. 

\begin{definition}[Match Pair *] \label{def:match*}
A match pair, $\langle\mathtt{R}, \mathtt{S}\rangle$, for a receive
$\mathtt{R}$ $=$ $(M_\mathtt{R},$ $\mathit{time}_\mathtt{R},$ $e_\mathtt{R},$ $\mathit{value}_\mathtt{R},$ $\mathit{time}_{\mathit{w}_\mathtt{R}})$ and a send $\mathtt{S}$ $=$ $(M_\mathtt{S},$ $\mathit{time}_\mathtt{S},$ $e_\mathtt{S},$ $\mathit{value}_\mathtt{S})$, corresponds to the constraints:
\begin{compactenum}
\item $M_{\mathtt{R}} = \mathit{time}_{\mathtt{S}}$
\item $M_{\mathtt{S}} = \mathit{time}_{\mathtt{R}}$
\item $e_{\mathtt{R}} = e_{\mathtt{S}}$
\item $src_\mathtt{R} = src_\mathtt{S}$
\item $\mathit{value}_{\mathtt{R}} = \mathit{value}_{\mathtt{S}}$ and
\item $\mathit{\textbf{time}}_{\mathtt{\textbf{S}}}\ = \ \mathit{\textbf{time}}_{\mathtt{\textbf{R}}} - \mathit{\textbf{1}}$
\end{compactenum}
\end{definition}

Intuitively, the consecutive order over a send and the matching receive is defined in the bold rule $6$ of \defref{def:match*}. Any resolved execution strictly alternates sends and receives as defined in \defref{def:alternate}. 

\begin{definition}[Strict Alternating]\label{def:alternate}
A set of sends, $S$, and a set of receives, $R$, are strictly alternated if and only if each send in $S$ immediately precedes the matching receive in $R$ and each receive in $R$ immediately follows the matching send in $S$.
\end{definition}

To further constrain the program order for zero buffer semantics, new rules are added: two sends are ordered on each process (rule 1*); on each process a send happens before a receive that is issued later (rule 8); and similarly, on each process a send happens before a barrier that is issued later (rule 9).

\paragraph*{Rule 1*} For each process, if there are sequential send
operations to any endpoints, say $\mathtt{S}$ and $\mathtt{S^\prime}$, then those
sends must follow program order: $\mathit{time}_\mathtt{S}$
$\prec_\mathtt{HB}$ $\mathit{time}_\mathtt{S^\prime}$.

\paragraph*{Rule 8} For each process, if there is a sequential order over a send operation and a receive operation, say $\mathtt{S}$ and $\mathtt{R}$, then they must follow program order: 
$\mathit{time}_\mathtt{S}$
$\prec_\mathtt{HB}$ $\mathit{time}_\mathtt{R}$.

\paragraph*{Rule 9} For each process, if there is a sequential order over a send operation and a barrier operation, say $\mathtt{S}$ and $\mathtt{B}$, then they must follow program order: 
$\mathit{time}_\mathtt{S}$
$\prec_\mathtt{HB}$ $\mathit{time}_\mathtt{B}$.

Rule $1^*$ replaces rule $1$ as zero buffer semantics do not allow a new send to be issued before the pending send is completed on a common process. Rule $8$ and rule $9$ constrain the sequential order over a send and a receive and the sequential order over a send and a barrier, respectively.

%: Correctness
\subsection{Correctness}
The prior work \cite{DBLP:conf/kbse/HuangMM13} proves the encoding technique is correct for wildcard receives. To be precise, the encoding is sound, meaning any satisfying schedule resolved by the encoding reflects an actual violating execution. It is also complete, meaning any actual violation can be resolved in a satisfying schedule by the encoding. 

To prove the new technique for MPI semantics is also correct, we rely on the existing soundness and completeness proofs. The soundness proof is consistent with the prior work for two aspects: 1) the program order is precisely constrained in the encoding; 2) any match pair used in a resolved satisfying schedule is valid. The deterministic receive operations and the collective operations for the new encoding do not violate both aspects. To be precise, the first aspect still exists because the program order for deterministic receive operations and collective operations are precisely defined in the new encoding. The second aspect is also true because the set of match pairs is given as input, which is not affected by deterministic receive operations and collective operations. 

The completeness proof of the new technique is similar to the prior work that uses the operational semantics to simulate the encoding during its operation to ensure that the two make identical conclusions. To prove the new encoding is complete, the operational semantics are extended to support deterministic receive operations and collective operations. A simulation of the extended operational semantics is then able to prove that any behavior of MPI semantics is encoded by the new technique in this paper. Because the soundness and the completeness for the new encoding are both proved, we conclude that the encoding is correct for MPI semantics. 

In addition to the soundness and completeness, the improved zero buffer encoding needs to prove the coverage of message communication. As shown earlier, the zero buffer encoding extends the rules for match pair and program order. The rules constrain the order over a send and the matching receive in a match pair to be consecutive. Based on this rule, the encoding assumes that sends and receives strictly alternate in any resolved execution without message non-determinism as in \defref{def:alternate}. This assumption is inspired by Threaded-C Bounded Model Checking (TCBMC) that assumes each lock operation and its paired unlock operation is ordered alternatingly in any execution \cite{DBLP:conf/cav/RabinovitzG05}. Similarly, the encoding in this paper detects assertion violations by only considering schedules that strictly alternate sends and receives in an execution. 

This strictly alternating assumption leads to an important proof obligation: how do we know this encoding covers all possible ways of message communication in zero buffer semantics? We prove a theorem later that asserts that strict alternating sequences of sends and receives cover the message communication that may occur in any execution under zero buffer semantics. Before that, we need to define a few terms.

\begin{definition}[Method Invocation]\label{def:invocation}
A invocation of a method, $M$, with a list of specific values of arguments, $(args\ \cdots)$, on process $P$, denoted as $P:M_\mathit{In}(args\ \cdots)$, is a event that occurs when $M$ is invoked. 
\end{definition}

\begin{definition}[Method Response]\label{def:response}
A response of a method, $M$, with a specific return value, $resp$, on process $P$, denoted as $P:M_\mathit{Re}(resp)$, is a event that occurs when $M$ returns. 
\end{definition}

Based on \defref{def:invocation} and \defref{def:response}, an operation is split into two events: invocation and response. The invocation asserts the issuing of an operation with concrete arguments. The response asserts the completion of an operation with a concrete return value. The next definition relies on both method invocation and method response. 

\begin{definition}[History]\label{def:history}
For an MPI program, let $\mathcal{H}$ be a history with a total order over method invocations and method responses for a set of send operations, $S$, a set of receive operations, $R$, and a set of barriers, $B$.
\end{definition}

Based on \defref{def:history}, we further define the legal history as follows.

\begin{definition}[Legal History]\label{def:legal}
A history, $\mathcal{H}$, for an MPI program is legal if the partial order over any pair of events in $\mathcal{H}$ is allowed by MPI semantics.
\end{definition}

A legal history defined in \defref{def:legal} represents a total order over events for an MPI program. A legal history only takes care of sends, receives and barriers because only they matter to message communication. In other words, the events in a legal history can be used to evaluate how messages may flow in a runtime system. Since the arguments are concrete in any method invocation and the return value is also concrete in any method response, a legal history corresponds to a precise resolution of message communication. To find a feasible message communication for a receive $R$, one only needs to search through the preceding events and find a send $S$ that matches the endpoints of $R$ and obeys the non-overtaking order for all the sends to the common destination endpoint. The legal history has to satisfy three conditions as shown in \defref{def:legal}. Informally, those conditions assert that the partial order of any pair of events, the point-to-point communication and the collective communication are all allowed by MPI semantics. In particular, a legal history may be only allowed by zero buffer semantics. In this case, it is called zero buffer legal history.

To prove the theorem later, we need to compare two legal histories for equivalency. The equivalency relation relies on the following definitions for projections. 

\begin{definition}[Projection To Process]\label{def:projection_process}
A projection of a legal history, $\mathcal{H}$, to a process, $P$, denoted as $\mathcal{H} | _P$, is a sequence of all the events on process $P$ in $\mathcal{H}$, such that the order over any pair of events in $\mathcal{H} | _P$ is identical as in $\mathcal{H}$.
\end{definition}

\begin{definition}[Projection To Receive Response]\label{def:projection_receive}
A projection of a legal history, $\mathcal{H}$, to the receive responses, $\mathit{set_R}$, denoted as $\mathcal{H} | _\mathit{set_R}$, is a sequence of receive responses in $\mathcal{H}$ such that the order over any pair of receive responses in $\mathcal{H} | _\mathit{set_R}$ is identical as in $\mathcal{H}$.
\end{definition}

Based on \defref{def:projection_process} and \defref{def:projection_receive}, a legal history can be further projected to the receive responses $\mathit{set_R}$ on process $P$ that produces a sequences of $\mathit{set_R}$ on process $P$. We use $\mathcal{H} | _{\mathit{set_R},P}$ to represent this projection. The equivalency relation relies on $\mathcal{H} | _{\mathit{set_R},P}$.

\begin{definition}[Equivalency Relation]\label{def:er}
Two legal histories for an identical MPI program, say $\mathcal{H}$ and $\mathcal{L}$ respectively, are equivalent, denoted as $\mathcal{H}$ $\sim$ $\mathcal{L}$, if and only if their projections to the receive responses, $\mathit{set_R}$, and each single process, $P$, $\mathcal{H} | _{\mathit{set_R},P}$ and $\mathcal{L} | _{\mathit{set_R},P}$ respectively, agree on the return values of $\mathit{set_R}$.
\end{definition}

We further use the notation $\not\sim$ for two legal histories $\mathcal{H}$ and $\mathcal{L}$ when they are not equivalent. The following lemma is essential to proving that \defref{def:er} is a valid equivalency relation.

\begin{lemma}[reflexivity, symmetry and transitivity]\label{lemma:rst}
The equivalency relation in \defref{def:er} is reflexive, symmetric and transitive.
\end{lemma}

\begin{proof}
For any legal histories $\mathcal{H}$, $\mathcal{L}$ and $\mathcal{T}$ for an identical MPI program,
\begin{compactenum}
\item $\mathcal{H}$ $\sim$ $\mathcal{H}$. The reflexivity is true because the projection $\mathcal{H}_{\mathit{set_R},P}$ to the receive responses $\mathit{set_R}$ on any process $P$ is identical to the projection of itself.
\item $\mathcal{H}$ $\sim$ $\mathcal{L}$ then $\mathcal{L}$ $\sim$ $\mathcal{H}$. The symmetry is also true because $\mathcal{H}$ $\sim$ $\mathcal{L}$ and $\mathcal{L}$ $\sim$ $\mathcal{H}$ both indicate that the projections $\mathcal{H}_{\mathit{set_R},P}$ and $\mathcal{L}_{\mathit{set_R},P}$ to the receive responses $\mathit{set_R}$ on any process $P$ agree on the return values of $\mathit{set_R}$. 
\item $\mathcal{H}$ $\sim$ $\mathcal{L}$ and $\mathcal{L}$ $\sim$ $\mathcal{T}$ then $\mathcal{H}$ $\sim$ $\mathcal{T}$. As for the transitivity, for all the receive responses $\mathit{set_R}$ on any process $P$ in the MPI program, $\mathcal{H}$ $\sim$ $\mathcal{L}$ indicates that the projections $\mathcal{H}_{\mathit{set_R},P}$ and $\mathcal{L}_{\mathit{set_R},P}$ to the receive responses $\mathit{set_R}$ on any process $P$ agree on the return values of $\mathit{set_R}$. Further, $\mathcal{L}$ $\sim$ $\mathcal{T}$ indicates that $\mathcal{L}_{\mathit{set_R},P}$ is also identical to $\mathcal{T}_{\mathit{set_R},P}$. Therefore, $\mathcal{H}_{\mathit{set_R},P}$ is identical to $\mathcal{T}_{\mathit{set_R},P}$. Since $P$ is an arbitrary process in the MPI program and $\mathit{set_R}$ do not change on $P$, thus $\mathcal{H}$ $\sim$ $\mathcal{T}$ is implied.
\end{compactenum}
\end{proof}

Based on \lemmaref{lemma:rst}, \defref{def:er} can be used to identify the equivalent classes among all legal histories for a given MPI program. 

\begin{definition}[Equivalent Class]\label{def:equal_class}
A set of legal histories, $\mathit{set_H}$, is a equivalent class, denoted as $\mathcal{E}$$(\mathit{set_H})$, if and only if,
\begin{compactenum}
\item for any two legal history, $L,T$ $\in$ $\mathit{set_H}$, $\mathcal{L}$ $\sim$ $\mathcal{T}$;
\item for any legal history, $L$ $\notin$ $\mathit{set_H}$, and any legal history, $T$ $\in$ $\mathit{set_H}$, $\mathcal{L}$ $\not\sim$ $\mathcal{T}$.
\end{compactenum}
\end{definition}

The following theorem further states that a representative exists for each equivalent class of legal histories under zero buffer semantics. 

\begin{theorem}
For any MPI program, each equivalent class of legal histories, $\mathcal{E}(\mathit{set_H})$, has a representative,$T$, that strictly alternates the sends, $\mathit{set_S}$ and the receives, $\mathit{set_R}$.
\end{theorem}

\begin{proof}
First, assume there is a legal history $\mathtt{L}$ $\in$ $\mathcal{E}(\mathit{set_H})$. The projection $\mathcal{L} | _\mathit{set_R}$ is a sequence of receive responses that reflects how messages are received in $\mathtt{L}$ for all processes. Since the message communication is precisely resolved in $\mathtt{L}$, each receive in $\mathit{R}$ is matched with a send in $\mathit{set_S}$. Based on $\mathcal{L}$ and $\mathcal{L} | _\mathit{set_R}$, a new sequence, $\mathcal{T}$, can be produced by two steps: 1) inserting the corresponding receive invocation immediately preceding each receive response; and 2) inserting the invocation and the response of the matching send immediately preceding each receive invocation. Based on those steps, $\mathcal{T}$ strictly alternates $\mathit{set_S}$ and $\mathit{set_R}$. Further, it obeys the conditions defined in \defref{def:legal} for two reasons. First, the consecutive order over a send and the matching receive in $\mathcal{L}$ still exists in $\mathcal{T}$; second, if the matching receive is issued earlier on process $P$, there is no way to execute any operation after the receive on process $P$ until it is matched with a send, therefore, postponing issuing the receive after the matching send does not violate the MPI semantics. Under zero buffer semantics, it is not possible to order a send and the matching receive other than these two situations. Notice that $\mathcal{T}$ is equivalent to $\mathtt{L}$ as they receive a common sequence of messages on each process. Therefore, for any existing legal history, the procedure above is able to find a equivalent legal history that strictly alternates sends and receives. 
\end{proof}
