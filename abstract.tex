Asynchronous message passing systems are fast becoming a common means for communication between devices.
%
One of the more common standards is the message passing interface (MPI).
%
A problem in any MPI program, intended or otherwise, is message race where a receive may match more than one send in a runtime system.
%
This non-determinism often leads to intermittent and unexpected behavior and is notorious for its difficulty to test or debug without exploring the full program state space. 
%
The use of barriers is able to synchronize the program computation. 
%
However, a side effect is its affection on the message communication.
%
Consequently, the semantics become more difficult to analyze formally. 
%
As both runtime semantics -- infinite-buffering (the message is copied into a runtime buffer on the API call) and zero-buffering (the message has no buffering) are imperatively requested by many MPI applications, both should be supported by a formal analysis as well. 
%
This paper presents an algorithm that encodes an MPI execution as a Satisfiability Modulo Theories (SMT) problem, which if satisfiable, yields a feasible schedule on the same trace, such that it resolves non-determinism in the MPI runtime in a way that it now fails user provided assertions.
%
This encoding correctly defines the point-to-point communication and the collective communication for MPI semantics.
%
It uses different rules to support both infinite-buffering and zero-buffering.
%
Results demonstrate that the SMT encoding runs fast and correctly for a set of benchmarks. 
%
As a result, the encoding scales well for programs with high levels of non-determinism in how sends and receives may potentially match.